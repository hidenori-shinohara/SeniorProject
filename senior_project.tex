\documentclass[psamsfonts]{amsart}

%-------Packages---------
\usepackage{amssymb,amsfonts}
\usepackage[all,arc]{xy}
\usepackage{enumerate}
\usepackage{mathrsfs}
\usepackage{theoremref}
\usepackage{graphicx}

%--------Theorem Environments--------
%theoremstyle{plain} --- default
\newtheorem{thm}{Theorem}[section]
\newtheorem{cor}[thm]{Corollary}
\newtheorem{prop}[thm]{Proposition}
\newtheorem{lem}[thm]{Lemma}
\newtheorem{conj}[thm]{Conjecture}
\newtheorem{quest}[thm]{Question}

\theoremstyle{definition}
\newtheorem{defn}[thm]{Definition}
\newtheorem{defns}[thm]{Definitions}
\newtheorem{con}[thm]{Construction}
\newtheorem{exmp}[thm]{Example}
\newtheorem{exmps}[thm]{Examples}
\newtheorem{notn}[thm]{Notation}
\newtheorem{notns}[thm]{Notations}
\newtheorem{addm}[thm]{Addendum}
\newtheorem{exer}[thm]{Exercise}

\theoremstyle{remark}
\newtheorem{rem}[thm]{Remark}
\newtheorem{rems}[thm]{Remarks}
\newtheorem{warn}[thm]{Warning}
\newtheorem{sch}[thm]{Scholium}

\makeatletter
\let\c@equation\c@thm
\makeatother
\numberwithin{equation}{section}

\bibliographystyle{plain}

%--------Meta Data: Fill in your info------
\title{Introduction to a matroid and its representability}

\author{Hidenori Shinohara}

\date{May 6, 2016}

\begin{document}


\begin{abstract}
This paper introduces a matroid. 
A matroid is a mathematical structure that abstracts the concept of linear independence.
The goal of this paper is to discuss the representability of a matroid with several examples and to introduce an important conjecture related to the representability of a matroid.
\end{abstract}

\maketitle

% \tableofcontents

% Chapter 1
% Introduction to Matroid
\section{Introduction to Matroid}

A matroid is a mathematical structure that abstracts the concept of linearly independence.
Linearly independent sets of vectors have a lot of good properties.
Interestingly, a lot of sets of other mathematical objects often share those properties.
For example, any subset of a linearly independent set of vectors is always linearly independent.
Consider a subset of edges of a graph that does not contain any cycle.
Then any subset of it does not contain any cycle.
The matroid theory is an attempt to mathematically formalize those properties and investigate in them.

\begin{defn}
A matroid $M = (E, \mathcal{I})$ is a pair such that $E$ is a finite set of elements, and $\mathcal{I}$ is a family of subsets of $E$ with the following properties:
\begin{itemize}
\item $\emptyset \in \mathcal{I}$
\item For any $A\in \mathcal{I}$, any subset of $A$ is in $\mathcal{I}$.
\item For any $A, B \in \mathcal{I}$ such that $\lvert A \rvert < \lvert B \rvert$, there always exists $x \in B - A$ such that $A \cup \{ x \} \in \mathcal{I}$.
\end{itemize}
$E$ is called the ground set, $\mathcal{I}$ is called the independent sets. A subset of $E$ is called independent if and only if it is in $\mathcal{I}$.
\end{defn}

In this paper, we will put our focus on a matroid with a finite ground set.

There are some basic matroids that are important in the following discussions.

We will start by introducing a column matroid. 
A column matroid is constructed from a matrix over a field $\mathbb{F}$.

\begin{defn}
Let a matrix $A$ with $m$ rows over $\mathbb{F}$ be given.
A column matroid $M$ of $A$ is a matroid with a ground set $\{ 1, 2, \cdots m \}$.
A subset of $E$ is independent in $M$ if and only if the set of column vectors corresponding to it is linearly independent.
\end{defn}

\begin{proof}
Prove it!
\end{proof}


\begin{defn}
A uniform matroid $U_{r, k}$ is a matroid such that $E = \{ 1, \cdots, k \}$ and $\mathcal{I} = \{ X \mid  X \subseteq E, \lvert X \rvert \leq r \}$.
\end{defn}

It is easy to see that a uniform matroid is indeed a matroid.



% Chapter 2
% Intro to representability
\section{Introduction to representability}

This chapter will introduce definitions and results that are necessary to discuss the representability.

More specifically, this chapter will introduce:
\begin{itemize}
\item Why we care about representability
  \begin{itemize}
  \item If all matroids are representable, we are just giving a vector space another name.
  \item Therefore, it is important that not all matroids are representable.
  \item Show some examples of unrepresentable matroids
  \item Now that we know the existence, the question is, which one?
  \end{itemize}
\item The mathematical definition of "representability"
\end{itemize}


\begin{defn}
Let $M = (E, \mathcal{I})$, $X \subseteq E$ be given.
$M \backslash X$ denotes a deletion of $X$ in $M$ and is defined to be $(E - X, \{ I \in \mathcal{I} \mid X \cap I = \emptyset \})$
\end{defn}

Note that $M \backslash e$ for some element $e \in E$ is equivalent to $M \backslash \{ e \}$.

\begin{thm}
Let $M = (E, \mathcal{I})$, $X \subseteq E$ be given.
$M \backslash X$ is indeed a matroid
\end{thm}

\begin{proof}
Let $M' = (E', \mathcal{I}') = M \backslash X$.
Since $\emptyset \in \mathcal{I}$ and $X \cap \emptyset = \emptyset$, $\emptyset \in \mathcal{I}$.
Let $I \in \mathcal{I}', J \subseteq I$. 
Since $I \in \mathcal{I}$, $J \in \mathcal{I}$.
Since $J \subseteq I$ and $I \cap X = \emptyset$, $J \cap X = \emptyset$.
Therefore, $J \in \mathcal{I}'$.
Let $A, B \in \mathcal{I}'$ such that $\lvert A \rvert < \lvert B \rvert$.
Since $A, B \in \mathcal{I}'$, we know that $A, B \in \mathcal{I}$.
Therefore, we can find $x \in B - A$ such that $(A \cup \{ x \}) \in \mathcal{I}$.
Since $(A \cup \{ x \}) \subseteq (A \cup B)$ and $X \cap A = X \cap B = \emptyset$, $X \cap (A \cup \{ x \}) = \emptyset$.
Therefore, $A \cup \{ x \} \in \mathcal{I}'$.
Thus we have found such $x \in B - A$ that $A \cup \{ x \} \in \mathcal{I}'$.
Since $M' = M \backslash X$ follows the three properties, it is indeed a matroid. 
\end{proof}

\begin{defn}
Let $M = (E, \mathcal{I})$, $e \in E$ be given.
$M / e$ denotes contraction of $M$ by $e$ and 
$M / e = \begin{cases}
      M \backslash e, \text{if $e$ is a loop},\\
      (E - \{ e \}, \{ I \in \mathcal{I} \mid e \notin I, (I \cup \{ e \}) \in \mathcal{I}\}), \text{ otherwise}.
         \end{cases}$
\end{defn}


\begin{thm}
Contraction by an element indeed generates a matroid.
\end{thm}

\begin{proof}
If $e$ is a loop, $M / e$ is obviously a matrod since we know that deletion always generates a matroid.
Suppose otherwise.
Let $\mathcal{I}'$ denote the independent sets of $M / e$.
First, $\emptyset \in \mathcal{I}, e \notin \emptyset$. Since $e$ is not a loop, $(\emptyset \cup \{ e \}) \in \mathcal{I}$.
Therefore, $\emptyset \in \mathcal{I}'$.
Let $I \in \mathcal{I}', J \subseteq I$.
Since $I \in \mathcal{I}$, $J \in \mathcal{I}$.
Since $e \notin I$, $e \notin J$.
Since $(I \cup \{ e \}) \in \mathcal{I}$ and $J \subseteq I$, $(J \cup \{ e \}) \in \mathcal{I}$.
Therefore, $J \in \mathcal{I}'$.
Let $A, B \in \mathcal{I}'$ such that $\lvert A \rvert < \lvert B \rvert$.
Let $A' = A \cup \{ e \}, B' = B \cup \{ e \}$.
Since $A, B \in \mathcal{I}'$, $A', B' \in \mathcal{I}$.
Since $e \notin A, e \notin B$, $\lvert A' \rvert < \lvert B' \rvert$.
Let $x \in B' - A'$ such that $A' \cup \{ x \} \in \mathcal{I}$.
Since $B' - A' = B - A$, $x \in B - A$.
For such $x$, we just showed that $A \cup \{ e \} \cup \{ x \} \in \mathcal{I}$.
Also, $x \neq e$ since $e \in A'$.
Therefore, $A \cup \{ x \} \in \mathcal{I}'$. 
Hence, we have found $x \in B - A$ such that $A \cup \{ x \} \in \mathcal{I}'$.
Since this follows three properties given in the definition, this is indeed a matroid.
Therefore, contraction by an element indeed generates a matroid.
\end{proof}


Here are a few simple yet useful results about deletion.

\begin{thm}
Let a matroid $M = (E, \mathcal{I}), X \subseteq E$ be given.
Let $A$ be an independent set in $M \backslash X$.
Then $A$ is independent in $M$.
\end{thm}

\begin{proof}
The independent sets of $M \backslash X$ is $\{ I \in \mathcal{I} \mid (I \cap X) = \emptyset \}$.
It is easy to see that it is a subset of $\mathcal{I}$.
Since $A$ is in the subset of $\mathcal{I}$, $A$ must be in $\mathcal{I}$.
\end{proof}

In other words, this means that deletion never "adds" a new element to the independent sets.


\begin{thm}
The deletion of a loop does not change the independent sets.
\end{thm}

\begin{proof}
Let $M = (E, \mathcal{I}), e \in E, \{ e \} \notin \mathcal{I}$.
The independent sets of $M \backslash e$ is $\{ I \in \mathcal{I} \mid (I \cap \{ e \}) = \emptyset \}$.
Since $\{ e \}$ is a loop, no independent set can contain $e$.
Therefore, the independent sets of $M \backslash e$ is identical to $\mathcal{I}$.
\end{proof}

Now that we have defined contraction of a matroid by an element, we can define contraction by a subset of a ground set.
\begin{defn}
Let $M = (E, \mathcal{I}), X = \{ x_1, \cdots, x_k \} \subseteq E$.
$M / X$ is defined to be $(((M/x_1)/x_2) \cdots)/x_k)$.
\end{defn}

It is not obvious that this is well-defined. 
In other words, it is not obvious that the order of contraction does not matter.
The following theorem shows that the order does not matter.

\begin{thm}
For any given matroid $M = (E, \mathcal{I})$,
$(M / e) / f = (M / f) / e$ for any $e \neq f \in E$.
\end{thm}

\begin{proof}
There are a few cases.
\begin{enumerate}

\item $e, f$ are both loops. \\
$(M / e) / f = (M \backslash e) / f$
Since deletion of a loop does not change the independent sets, $f$ is a loop in $(M \backslash e)$.
Therefore, $(M / e) / f = (M \backslash e) \backslash f$.
Again, deletion of $f$ does not change the independent sets since $f$ is a loop in $(M / e)$.
Therefore, we have $(M / e) / f = (E - \{ e, f \}, \mathcal{I})$.
By symmetry, $(M / e) / f = (M / f) / e$.

\item 
One of $e, f$ is a loop, and the other one is not. \\
Without loss of generality, assume $e$ is a loop.
$(M / e) / f = (M \backslash e) / f$.
Since deletion of a loop does not change the independent set, the independent set of $(M / e)$ is $\mathcal{I}$.
Therefore, the independent set of $(M / e) / f$ is $\mathcal{I}' = \{ I \in \mathcal{I} \mid f \notin I, (I \cup \{ f \}) \in \mathcal{I} \}$.
On the other hand, it is easy to see that $\mathcal{I}'$ is identical to the independent sets of $(M / f)$.
Since contraction by an element does not add new elements to the independent sets, $e$ is a loop in $(M / f)$.
Since deletion by a loop does not change the independent sets, the independent sets of $(M / f) / e$ is $\mathcal{I}'$.
Now we confirmed that $(M / e) / f$ and $(M / f) / e$ have the same independent sets.
Therefore, $(M / e) / f = (M / f) / e$.

\item Neither of them is a loop, and $\{ e, f \} \in \mathcal{I}$. \\
The independent set of $M / e$ is $\mathcal{I}' = \{ I \in \mathcal{I} \mid e \notin I, (I \cup \{ e \}) \in \mathcal{I} \}$.
Since $\{ e, f \} \in \mathcal{I}$, $\{ f \} \in \mathcal{I}'$. 
Therefore, $f$ is not a loop in $M / e$.
Hence, the independent sets of $(M / e) / f$ is $\mathcal{I}'' = \{ I \in \mathcal{I}' \mid f \notin I, (I \cup \{ f \}) \in \mathcal{I}' \}$.
$\mathcal{I}''$ is actually equivalent to $S = \{ I \in \mathcal{I} \mid e \notin I, f \notin I, (I \cup \{ e, f \}) \in \mathcal{I} \}$.
We can prove $\mathcal{I}'' = S$ by starting to show that $\mathcal{I}'' \subseteq S$.
Let $I \in \mathcal{I}''$.
Since $I$ is an independent set of $(M / e) / f$, we know that $e, f \notin I$.
Since $(I \cup \{ f \}) \in \mathcal{I}'$, we also know that $((I \cup \{ f \}) \cup \{ e \}) \in \mathcal{I}$.
Therefore, $(I \cup \{ e, f \}) \in \mathcal{I}$.
Thus $I \in S$, and $\mathcal{I}'' \subseteq S$.
Now, we want to show that $S \subseteq \mathcal{I}''$. 
Let $I \in S$. By the definition of $S$, we know that $e, f \notin I$.
Since $(I \cup \{ e, f \}) \in \mathcal{I}$, we know that $(I \cup \{ e \}) \in \mathcal{I}$.
Since $((I \cup \{ f \}) \cup \{ e \}) \in \mathcal{I}$ and $e \notin (I \cup \{ f \})$, we know that $(I \cup \{ f \}) \in \mathcal{I}'$.
Since $f \notin I$ and $(I \cup \{ f \}) \in \mathcal{I}'$, $I \in \mathcal{I}''$.
Hence, $S \subseteq \mathcal{I}''$.

Combining these two results, we know that $S = \mathcal{I}''$.
By the symmetry, $(M / e) / f$ and $(M / f) / e$ have the same independent sets.
Therefore $(M / e) / f = (M / f) / e$.

\item Neither of $e, f$ is a loop, but $\{ e, f \} \notin \mathcal{I}$.\\
The independent set of $M / e$ is $\mathcal{I}' = \{ I \in \mathcal{I} \mid e \notin I, (I \cup \{ e \}) \in \mathcal{I} \}$.
Since $\{ e, f \} \notin \mathcal{I}$, $f$ is a loop in $M / e$.
Therefore, $(M / e) / f = (M / e) \backslash f$.
Since the deletion of a loop does not change the independent sets, the independent sets of $(M / e) / f$ is $\mathcal{I}'$.
By applying the same argument, the independent set of $(M / f) / e$ is $\mathcal{I}'' = \{ I \in \mathcal{I} \mid f \notin I, (I \cup \{ f \} ) \in \mathcal{I} \}$.
We want to show that $\mathcal{I}'  = \mathcal{I}''$.
By the symmetry, it suffices to show that $\mathcal{I}' \subseteq \mathcal{I}''$.
Let $I \in \mathcal{I}'$.
Since $\{ e, f \}$ is dependent, $f \notin I$. (Otherwise, $I \cup \{ e \}$ would be dependent.)
Since both $(I \cup \{ e \})$ and $\{ f \}$ are independent, we can grow $\{ f \}$ by adding elements from $(I \cup \{ e \})$ until they have the same size.
Since $\{ e, f\}$ is dependent, we never add $e$.
In other words, we add every element from $\mathcal{I}$.
It means that $I \cup \{ f \}$ is independent.
Therefore, $I \in \mathcal{I}''$, and thus $\mathcal{I}' \subseteq \mathcal{I}''$.
By symmetry, $\mathcal{I}'' \subseteq \mathcal{I}'$.
Therefore, $\mathcal{I}' = \mathcal{I}''$.
\end{enumerate}

Therefore, in any case, $(M / e) / f = (M / f) / e$.
\end{proof}


List of topics I could discuss further:
\begin{itemize}
\item Maybe talk about dual matroid
  \begin{itemize}
    \item redefine contraction/deletion in terms of dual
  \end{itemize}

\item More properties about contraction deletion
  \begin{itemize}
    \item $(M / e) \backslash f = (M \backslash f) / e$ ?
    \item Therefore, any minor can be represented as $(M \backslash A) / B$
  \end{itemize}
\end{itemize}


\subsection{What do contraction and deletion mean in graphs and vector spaces?}

\subsection{Why do these matter?}
Because if a matroid is representable over some field $\mathbb{F}$, its minor is always representable over $\mathbb{F}$.

\subsection{The introduction of Rota's conjecture}



% Chapter 3
% more discussion on representability
\section{More discussion on matroid representability}
This chapter will introduce a new concept, a matroid minor, which is crucial when discussing the matroid representability.

In order to introduce a matroid minor, we first need to introduce two operations on matroids: deletion and contraction.

\begin{defn}
Let $M = (E, \mathcal{I})$, $X \subseteq E$ be given.
$M \backslash X$ denotes a deletion of $X$ in $M$ and is defined to be $(E - X, \{ I \in \mathcal{I} \mid X \cap I = \emptyset \})$
\end{defn}

Note that $M \backslash e$ for some element $e \in E$ is equivalent to $M \backslash \{ e \}$.

\begin{thm}
Let $M = (E, \mathcal{I})$, $X \subseteq E$ be given.
$M \backslash X$ is indeed a matroid
\end{thm}

\begin{proof}
Let $M' = (E', \mathcal{I}') = M \backslash X$.
Since $\emptyset \in \mathcal{I}$ and $X \cap \emptyset = \emptyset$, $\emptyset \in \mathcal{I}$.
Let $I \in \mathcal{I}', J \subseteq I$. 
Since $I \in \mathcal{I}$, $J \in \mathcal{I}$.
Since $J \subseteq I$ and $I \cap X = \emptyset$, $J \cap X = \emptyset$.
Therefore, $J \in \mathcal{I}'$.
Let $A, B \in \mathcal{I}'$ such that $\lvert A \rvert < \lvert B \rvert$.
Since $A, B \in \mathcal{I}'$, we know that $A, B \in \mathcal{I}$.
Therefore, we can find $x \in B - A$ such that $(A \cup \{ x \}) \in \mathcal{I}$.
Since $(A \cup \{ x \}) \subseteq (A \cup B)$ and $X \cap A = X \cap B = \emptyset$, $X \cap (A \cup \{ x \}) = \emptyset$.
Therefore, $A \cup \{ x \} \in \mathcal{I}'$.
Thus, we have found such $x \in B - A$ that $A \cup \{ x \} \in \mathcal{I}'$.
Since $M' = M \backslash X$ follows the three properties, it is indeed a matroid. 
\end{proof}

\begin{defn}
Let $M = (E, \mathcal{I})$, $e \in E$ be given.
$M / e$ denotes contraction of $M$ by $e$ and 
$M / e = \begin{cases}
      M \backslash e, \text{if $e$ is a loop},\\
      (E - \{ e \}, \{ I \in \mathcal{I} \mid e \notin I, (I \cup \{ e \}) \in \mathcal{I}\}), \text{ otherwise}.
         \end{cases}$
\end{defn}


\begin{thm}
Contraction by an element indeed generates a matroid.
\end{thm}

\begin{proof}
If $e$ is a loop, $M / e$ is obviously a matroid since we know that deletion always generates a matroid.
Suppose otherwise.
Let $\mathcal{I}'$ denote a family of independent sets of $M / e$.
First, $\emptyset \in \mathcal{I}, e \notin \emptyset$. Since $e$ is not a loop, $(\emptyset \cup \{ e \}) \in \mathcal{I}$.
Therefore, $\emptyset \in \mathcal{I}'$.
Let $I \in \mathcal{I}', J \subseteq I$.
Since $I \in \mathcal{I}$, $J \in \mathcal{I}$.
Since $e \notin I$, $e \notin J$.
Since $(I \cup \{ e \}) \in \mathcal{I}$ and $J \subseteq I$, $(J \cup \{ e \}) \in \mathcal{I}$.
Therefore, $J \in \mathcal{I}'$.
Let $A, B \in \mathcal{I}'$ such that $\lvert A \rvert < \lvert B \rvert$.
Let $A' = A \cup \{ e \}, B' = B \cup \{ e \}$.
Since $A, B \in \mathcal{I}'$, $A', B' \in \mathcal{I}$.
Since $e \notin A, e \notin B$, $\lvert A' \rvert < \lvert B' \rvert$.
Let $x \in B' - A'$ such that $A' \cup \{ x \} \in \mathcal{I}$.
Since $B' - A' = B - A$, $x \in B - A$.
For such $x$, we just showed that $A \cup \{ e \} \cup \{ x \} \in \mathcal{I}$.
Also, $x \neq e$ since $e \in A'$.
Therefore, $A \cup \{ x \} \in \mathcal{I}'$. 
Hence, we have found $x \in B - A$ such that $A \cup \{ x \} \in \mathcal{I}'$.
Since this follows three properties given in the definition, this is indeed a matroid.
Therefore, contraction by an element indeed generates a matroid.
\end{proof}


Here are a few simple yet useful results about deletion.

\begin{thm}
Let a matroid $M = (E, \mathcal{I}), X \subseteq E$ be given.
Let $A$ be an independent set in $M \backslash X$.
Then $A$ is independent in $M$.
\end{thm}

\begin{proof}
A family of independent sets of $M \backslash X$ is $\{ I \in \mathcal{I} \mid (I \cap X) = \emptyset \}$.
It is easy to see that it is a subset of $\mathcal{I}$.
Since $A$ is in the subset of $\mathcal{I}$, $A$ must be in $\mathcal{I}$.
\end{proof}

In other words, this means that deletion never "adds" a new element to a family of independent sets.


\begin{thm}
The deletion of a loop does not change a family of independent sets.
\end{thm}

\begin{proof}
Let $M = (E, \mathcal{I}), e \in E, \{ e \} \notin \mathcal{I}$.
A family of independent sets of $M \backslash e$ is $\{ I \in \mathcal{I} \mid (I \cap \{ e \}) = \emptyset \}$.
Since $\{ e \}$ is a loop, no independent set can contain $e$.
Therefore, a family of independent sets of $M \backslash e$ is identical to $\mathcal{I}$.
\end{proof}

Now that we have defined contraction of a matroid by an element, we can define contraction by a subset of a ground set.
\begin{defn}
Let $M = (E, \mathcal{I}), X = \{ x_1, \cdots, x_k \} \subseteq E$.
$M / X$ is defined to be $(((M/x_1)/x_2) \cdots)/x_k)$.
\end{defn}

It is not obvious that this is well-defined. 
In other words, it is not obvious that the order of contraction does not matter.
The following theorem shows that the order does not matter.

\begin{thm}
For any given matroid $M = (E, \mathcal{I})$,
$(M / e) / f = (M / f) / e$ for any $e \neq f \in E$.
\end{thm}

\begin{proof}
There are a few cases.
\begin{enumerate}

\item $e, f$ are both loops. \\
$(M / e) / f = (M \backslash e) / f$
Since deletion of a loop does not change a family of independent sets, $f$ is a loop in $(M \backslash e)$.
Therefore, $(M / e) / f = (M \backslash e) \backslash f$.
Again, deletion of $f$ does not change a family of independent sets since $f$ is a loop in $(M / e)$.
Therefore, we have $(M / e) / f = (E - \{ e, f \}, \mathcal{I})$.
By symmetry, $(M / e) / f = (M / f) / e$.

\item 
One of $e, f$ is a loop, and the other one is not. \\
Without loss of generality, assume $e$ is a loop.
$(M / e) / f = (M \backslash e) / f$.
Since deletion of a loop does not change a family of independent set, a family of independent set of $(M / e)$ is $\mathcal{I}$.
Therefore, a family of independent sets of $(M / e) / f$ is $\mathcal{I}' = \{ I \in \mathcal{I} \mid f \notin I, (I \cup \{ f \}) \in \mathcal{I} \}$.
On the other hand, it is easy to see that $\mathcal{I}'$ is identical to a family of independent sets of $(M / f)$.
Since contraction by an element does not add new elements to a family of independent sets, $e$ is a loop in $(M / f)$.
Since deletion by a loop does not change a family of independent sets, a family of independent sets of $(M / f) / e$ is $\mathcal{I}'$.
Now we confirmed that $(M / e) / f$ and $(M / f) / e$ have the same family of independent sets.
Therefore, $(M / e) / f = (M / f) / e$.

\item Neither of them is a loop, and $\{ e, f \} \in \mathcal{I}$. \\
A family of independent set of $M / e$ is $\mathcal{I}' = \{ I \in \mathcal{I} \mid e \notin I, (I \cup \{ e \}) \in \mathcal{I} \}$.
Since $\{ e, f \} \in \mathcal{I}$, $\{ f \} \in \mathcal{I}'$. 
Therefore, $f$ is not a loop in $M / e$.
Hence, a family of independent sets of $(M / e) / f$ is $\mathcal{I}'' = \{ I \in \mathcal{I}' \mid f \notin I, (I \cup \{ f \}) \in \mathcal{I}' \}$.
$\mathcal{I}''$ is actually equivalent to $S = \{ I \in \mathcal{I} \mid e \notin I, f \notin I, (I \cup \{ e, f \}) \in \mathcal{I} \}$.
We can prove $\mathcal{I}'' = S$ by starting to show that $\mathcal{I}'' \subseteq S$.
Let $I \in \mathcal{I}''$.
Since $I$ is an independent set of $(M / e) / f$, we know that $e, f \notin I$.
Since $(I \cup \{ f \}) \in \mathcal{I}'$, we also know that $((I \cup \{ f \}) \cup \{ e \}) \in \mathcal{I}$.
Therefore, $(I \cup \{ e, f \}) \in \mathcal{I}$.
Thus $I \in S$, and $\mathcal{I}'' \subseteq S$.
Now, we want to show that $S \subseteq \mathcal{I}''$. 
Let $I \in S$. By the definition of $S$, we know that $e, f \notin I$.
Since $(I \cup \{ e, f \}) \in \mathcal{I}$, we know that $(I \cup \{ e \}) \in \mathcal{I}$.
Since $((I \cup \{ f \}) \cup \{ e \}) \in \mathcal{I}$ and $e \notin (I \cup \{ f \})$, we know that $(I \cup \{ f \}) \in \mathcal{I}'$.
Since $f \notin I$ and $(I \cup \{ f \}) \in \mathcal{I}'$, $I \in \mathcal{I}''$.
Hence, $S \subseteq \mathcal{I}''$.

Combining these two results, we know that $S = \mathcal{I}''$.
By the symmetry, $(M / e) / f$ and $(M / f) / e$ have the same family of independent sets.
Therefore $(M / e) / f = (M / f) / e$.

\item Neither of $e, f$ is a loop, but $\{ e, f \} \notin \mathcal{I}$.\\
A family of independent set of $M / e$ is $\mathcal{I}' = \{ I \in \mathcal{I} \mid e \notin I, (I \cup \{ e \}) \in \mathcal{I} \}$.
Since $\{ e, f \} \notin \mathcal{I}$, $f$ is a loop in $M / e$.
Therefore, $(M / e) / f = (M / e) \backslash f$.
Since the deletion of a loop does not change a family of independent sets, a family of independent sets of $(M / e) / f$ is $\mathcal{I}'$.
By applying the same argument, a family of independent set of $(M / f) / e$ is $\mathcal{I}'' = \{ I \in \mathcal{I} \mid f \notin I, (I \cup \{ f \} ) \in \mathcal{I} \}$.
We want to show that $\mathcal{I}'  = \mathcal{I}''$.
By the symmetry, it suffices to show that $\mathcal{I}' \subseteq \mathcal{I}''$.
Let $I \in \mathcal{I}'$.
Since $\{ e, f \}$ is dependent, $f \notin I$. (Otherwise, $I \cup \{ e \}$ would be dependent.)
Since both $(I \cup \{ e \})$ and $\{ f \}$ are independent, we can grow $\{ f \}$ by adding elements from $(I \cup \{ e \})$ until they have the same size.
Since $\{ e, f\}$ is dependent, we never add $e$.
In other words, we add every element from $\mathcal{I}$.
It means that $I \cup \{ f \}$ is independent.
Therefore, $I \in \mathcal{I}''$, and thus $\mathcal{I}' \subseteq \mathcal{I}''$.
By symmetry, $\mathcal{I}'' \subseteq \mathcal{I}'$.
Therefore, $\mathcal{I}' = \mathcal{I}''$.
\end{enumerate}

Therefore, in any case, $(M / e) / f = (M / f) / e$.
\end{proof}


%\item Maybe talk about dual matroid
%  \begin{itemize}
%    \item redefine contraction/deletion in terms of dual
%  \end{itemize}

Now that we have defined contraction and deletion, we can define a \textit{minor} of a matroid.

\begin{defn}
A minor of a matroid is a matroid that can be obtained by some (possibly zero) a number of contraction and deletion.
\end{defn}

Therefore, most matroids have more than one minor.
To define the matroid minor more concisely, we will prove the following theorem.


\begin{thm}
Let a matroid $M = (E, \mathcal{I})$ and $e \neq f \in E$ be given.
Then $(M / e) \backslash f = (M \backslash f) / e$.
\end{thm}

\begin{proof}
There are a few cases.
\begin{enumerate}
\item
Both $e$ and $f$ are loops.
This case is easy to prove since neither contraction nor deletion add any new elements to a family of independent sets.
In other words, $f$ is a loop in $(M / e)$ and $e$ is a loop in $(M \backslash f)$.
$(M / e) \backslash f = (M \backslash e) \backslash f = (M \backslash f) \backslash e = (M \backslash f) / e$
\item
$e$ is a loop, but $f$ is not a loop.
$(M / e) \backslash f = (M \backslash e) \backslash f$ since $e$ is a loop.
$(M \backslash f) / e = (M \backslash f) \backslash e$ since $e$ is a loop in $(M \backslash f)$.
We know that the order of deletion does not matter, so they are equivalent.
\item
$e$ is not a loop, but $f$ is a loop.
Since the deletion of $f$ does not change a family of independent sets,
$M$ and $(M \backslash f)$ both have the same independent sets, although their ground sets are not identical.
Therefore, $(M / e)$ and $(M \backslash f) / e$ have the same independent sets from the definition of contraction.
\item
Neither $e$ nor $f$ is a loop.
Let $X \subseteq E - \{e, f\}$.
$X$ is independent in $(M / e) \backslash f$ if and only if $X$ is independent in $(M / e)$.
$X$ is independent in $(M / e)$ if and only if $(X \cup \{ e \})$ is independent in $M$.
On the other hand, $X$ is independent in $(M \backslash f) / e$ if and only if $X \cup \{ e \}$ is independent in $(M \backslash f)$.
$X \cup \{ e \}$ is independent in $(M \backslash f)$ if and only if $X \cup \{ e \}$ is independent in $M$.
Therefore, $X$ is independent in $(M / e) \backslash f$ if and only if $X$ is independent in $(M \backslash f) / e$.
\end{enumerate}
Therefore, in every case, $(M / e) \backslash f$ is identical to $(M \backslash f) / e$.
\end{proof}

By this theorem, we know that any series of operations can be expressed as $(M / A) \backslash B$ where $A, B$ are disjoint subsets of $E$.
Therefore, the following definition is equivalent to the previous definition.

\begin{defn}
Let $M = (E, \mathcal{I})$ be given.
Let $A, B$ be disjoint subsets of $E$.
Then a matroid $(M / A) \backslash B$ is called a minor of $M$.
\end{defn}

Moreover, if $A \cup B \neq \emptyset$, we call $(M / A) \backslash B$ a \textit{proper minor}.

Of course, we could have defined a minor as $(M \backslash A) / B$ instead of $(M / A) \backslash B$.


\subsection{What do contraction and deletion mean in graphs and vector spaces?}
Deletion of a vector in the vector space is simply removing such a vector from the set.
Contraction, however, is more complicated.
In vector spaces, contraction can be considered as projection to its orthogonal vector as showin in the figure 2.
In graphs, deletion of an edge simply removes an edge. Contraction of an edge combines the two nodes that the edge connects as shown in the figure 3.

\begin{figure}
  \centering
    \includegraphics[width=0.8\textwidth,natwidth=610,natheight=642]{vectors.png}
    \caption{Contraction by a vector A}
  \label{fig:test}
\end{figure}

\begin{figure}
  \centering
    \includegraphics[width=0.8\textwidth,natwidth=610,natheight=642]{graphs.png}
    \caption{Contraction by an edge 3}
  \label{fig:test}
\end{figure}



\subsection{Why do these matter?}
The discussion of contraction and deletion is very important when discussing the representability of matroids since if a matroid is representable over some field $\mathbb{F}$, its minor is always representable over $\mathbb{F}$.

\begin{thm}
Let a matroid $M = (E, \mathcal{I})$ such that it is representable over $\mathbb{F}$.
Any minor of $M$ is representable over $\mathbb{F}$.
\end{thm}

\begin{proof}
It suffices to show that $M / e$ and $M \backslash e$ are both representable over $\mathbb{F}$ for any $e \in E$.
Let $r$ be a rank of $M$, $n = \lvert E \rvert$.
Let $e \in E$ be given.
Let $A = \begin{pmatrix}u_1 u_2 \cdots u_n\end{pmatrix} \in \mathbb{F}^{r \times n}$ be a matrix such that the column matroid of $M$ is isomorphic to $A$.
Without loss of generality, we can assume $E = \{ 1, 2, \cdots, n \}$ and $e = n$.
\begin{enumerate}
\item 
  First, we prove the case of deletion.
  We claim that $M \backslash e$ is isomorphic to the column matroid of $\begin{pmatrix} u_1 u_2 \cdots u_{n-1} \end{pmatrix}$.
  We prove so by comparing the independent sets of each matroid.
  By definition, $M \backslash e = (E - \{ e \}, \{ I \in \mathcal{I} \mid e \notin I \})$.
  Let $U = \{ u_{i_1}, u_{i_2}, \cdots, u_{i_k} \}$ be a subset of $\{ u_1, u_2, \cdots, u_{n-1} \}$.
  We want to show that $U$ is linearly independent if and only if $\{ i_1, i_2, \cdots i_k \}$ is linearly independent in $M \backslash e$.
  Suppose $U$ is linearly independent. 
  Then $\{ i_1, i_2, \cdots i_k \}$ is independent in the column matroid of $A$.
  Since $\{ i_1, i_2, \cdots i_k \}$ is in $\mathcal{I}$ and does not contain $e = n$, it is independent in $M \backslash e$ as well.
  Suppose $U$ is linearly dependent. 
  Then $\{ i_1, i_2, \cdots i_k \}$ is dependent in the column matroid of $A$.
  Since $\{ i_1, i_2, \cdots i_k \}$ is not in $\mathcal{I}$ it is dependent in $M \backslash e$ as well.
  Therefore, $M \backslash e$ is representable over $\mathbb{F}$.
\item
  Next, we prove the case of contraction.
  First, assume $e$ is a loop.
  Then the corresponding column of $A$ must be a zero vector.
  It is easy to see that the removal of the corresponding column will yield a matrix whose column vector is isomorphic to $M / e = M \backslash e$.
  Now, assume that $e$ is not a loop.
  Then the corresponding column of $A$ must not be a zero vector.
  We prove the proposition by comparing the independent sets of each matroid.
  By definition, $M / e = (E - \{ e \}, \{ I \in \mathcal{I} \mid e \notin I, (e \cup I) \in \mathcal{I}\})$.
  We claim that $M / e$ is isomorphic to the column matroid of $B \in \mathbb{F}^{r \times n-1}$, 
  where $i$th column of $B$, $b_i$, is $\displaystyle u_i - \frac{u_i \cdot u_n}{u_n \cdot u_n} u_n$.
  (This makes sense since we are assuming that $u_n$ is not a zero vector.)
  Let $X = \{ i_1, i_2, \cdots, i_k \} \subseteq E - \{ n \}$ be given. \\
  $X$ is independent in $M / e$ \\
  $\iff$ $X \cup \{ n \}$ is independent in $M$ \\
  $\iff$ $\{ i_1, i_2, \cdots, i_k, n \}$ is independent in $M$ \\
  $\iff$ $\{ u_{i_1}, u_{i_2}, \cdots, u_{i_k}, u_{n} \}$ is linearly independent \\
  We are going to take a close look at this set of vectors.
  Let $c_1, c_2, \cdots, c_k, c \in \mathbb{F}$ be given such that $c_1 u_{i_1} + \cdots + c_k u_{i_k} + c u_n = 0$.\\
  \begin{align*}\displaystyle c_1 b_{i_1} + \cdots + c_k b_{i_k}
  &= \sum_{j=1,\cdots,k} c_j b_{i_j}  \\
  &= c_1 u_{i_1} + c_2 u_{i_2} + \cdots + c_k u_{i_k} - \sum_{j=1,\cdots,k} c_j \frac{u_i \cdot u_n}{u_n \cdot u_n} u_n  \\
  &= -c u_n - \sum_{j=1,\cdots,k} c_j \frac{u_i \cdot u_n}{u_n \cdot u_n} u_n \\
  &= -\Big(c + \sum_{j=1,\cdots,k} c_j \frac{u_i \cdot u_n}{u_n \cdot u_n}\Big)u_n \\
  &= -\frac{\Big(c{u_n \cdot u_n} + \sum_{j=1,\cdots,k} c_j (u_i \cdot u_n)\Big)}{u_n \cdot u_n}u_n \\
  &= -\frac{(c_1 u_1 + c_2 u_2 + \cdots + c_k u_k + c u_n) \cdot u_n}{u_n \cdot u_n}u_n \\
  &= 0.\end{align*}
  Therefore, $\{ u_{i_1}, u_{i_2}, \cdots, u_{i_k}, u_{n} \}$ is linearly independent \\
  $\iff$ $\{ b_{i_1}, b_{i_2}, \cdots, b_{i_k} \}$ is linearly independent \\
  $\iff$ $\{ i_1, i_2, \cdots, i_k \}$ is independent in the column matroid of $B$. \\
  Therefore, the column matroid of $B$ is isomorphic to $M / e$.
\end{enumerate}
Hence, we have proved that any minor of $M$ is always representable.
\end{proof}

However, neither the converse nor the inverse of this theorem is always true.
Any matroid has a representable minor since $U_{0, k}$ is a minor of any matroid.
Also, $U_{2, 4}$ is not a binary matroid, but any minor of it only contains at most 3 elements, so we know that any minor of $U_{2,4}$ is regular by the theorem.
Rota's conjecture is about unrepresentable matroids any of whose minor is representable.
It will be discussed in the next chapter.




% Chapter 4
% discussion on Rota's conjecture
\section{The introduction to Rota's Conjecture}
This chapter introduces Rota's Conjecture and an outline of the proof.

\begin{conj}[Rota's Conjecture]
For each finite field $\mathbb{F}$, there are, up to isomorphism, only finitely many excluded minors for the class of $\mathbb{F}$-representable matroids.
\end{conj}

In other words, given a finite field $\mathbb{F}$, let $F$ be a family of all matroids that are representable over $\mathbb{F}$.
By the theorem from the previous chapter, we know that $F$ is minor-closed, i.e., any minor of any element in $F$ is in $F$.
Then, an excluded minor is a matroid $M \notin F$ such that any proper minor of $M$ is in $F$.
The conjecture states that for each finite field $\mathbb{F}$, there are only finitely many excluded minors.

First, we will introduce some results related to excluded minors.

\begin{thm}
The Fano matroid is an excluded minor for the class of $\mathbb{F}$-representable matroids if the characteristic of $\mathbb{F}$ is not 2.
\end{thm}
\begin{proof}
Let $M$ be the Fano matroid.
We know from the previous chapter that $M$ is not representable in a field if the characteristic of the field is not 2.
Therefore, it suffices to show that any minor of $M$ is representable in those fields.
Since any minor of a representable matroid is always representable, it suffices to show that $M / e$ and $M \backslash e$ are representable for any $e$ in those fields.
It is obvious that $M / 1$, $M / 2$ and $M / 3$ are isomorphic to each other by the symmetry of the Fano plane.
For the same reason, each matroid in each of the following sets is isomorphic to other matroids in the same set.
$\{ M \backslash 1, M \backslash 2, M \backslash 3 \}, \{ M / 4, M / 5, M / 6 \}, \{ M \backslash 4, M \backslash 5, M \backslash 6 \}$.
If we swap 1 with 4, 3 with 5, the Fano place still gives the identical matroid.
That implies that deletion or contraction of any of 1, 2, 3, 4, 5, 6 always gives isomorphic matroids.
Similarly, if we swap 6 with 7, 3 with 5, the Fano place still gives the identical matroid.
That implies that deletion or contraction of any element always gives isomorphic matroids.
Now that we have shown the symmetry, it suffices to show that $M / 7$ and $M \backslash  7$ are both representable in those fields.
$M \backslash 7$ is isomorphic to the column matroid of the following matrix:
$\begin{pmatrix}
1 & 0 & 0 & 1 & 0 & 1 \\
0 & 1 & 0 & 1 & 1 & 0 \\
0 & 0 & 1 & 0 & 1 & -1 \\
\end{pmatrix}$, where -1 is the additive inverse of 1.
Similarly, $M / 7$ is isomorphic to the column matroid of the following matrix:
$\begin{pmatrix}
1 & 0 & 1 & 1 & 1 & 0 \\
0 & 1 & 1 & 1 & 0 & 1
\end{pmatrix}$.
\end{proof}

Rota's conjecture has been proved for some finite fields.
For example, $U_{2, 4}$ is the only excluded minor of $GF(2)$, and the proof can be found in \cite{lec9}.
In $GF(3)$, there are 4 excluded minors, which are $F_7, F_7^\ast, U_{2, 5}, U_{3, 5}$.\cite{lec9}
In $GF(4)$, there are 7 excluded minors, which are $U_{2, 6}, U_{4, 6}, P_6, F_7^-, (F_7^-)^\ast, P_8, P_8''$.\cite{gf4}

Here is a sketch of proof that $U_{2, 4}$ is the only excluded minor of $GF(2)$.
The complete proof can be found in \cite{lec9}.

First, we will assume the following lemmas. Proofs of them can be found in \cite{lec9}.

\begin{lem}
Let $B$ be a basis of a matroid $M = (S, \mathcal{I})$, and $e \notin B$.
Then $B \cup \{ e \}$ contains a unique circuit.
A circuit is a minimal dependent subset.
\end{lem}

\begin{lem}
Let $M, N$ be distinct matroids on the same ground set $S$.
Suppose that there exists $B \subseteq S$ such that
\begin{itemize}
\item $B$ is a basis of $M$ and $N$.
\item There is no $X$ such that $\lvert B \Delta X \rvert = 2$ and $X$ is a basis of exactly one of $M$ and $N$.
\end{itemize}
Then $M$ or $N$ has a $U_{2, 4}$ minor.
\end{lem}

\begin{thm}[Tutte's Theorem]
$U_{2, 4}$ is the only excluded minor of $GF(2)$.
In other words, a matroid $M$ is representable over $GF(2)$ if and only if $U_{2, 4}$ is not $M$'s minor.
\end{thm}

\begin{proof}
We proved earlier that $U_{2, 4}$ is not representable over $GF(2)$.
Since any proper minor of $U_{2, 4}$ contains at most 3 elements, we know that any proper minor of $U_{2, 4}$ is representable over $GF(2)$.
Therefore, it suffices to prove that any matroid $M$ that is not representable over $GF(2)$ has a $U_{2, 4}$ minor.
Let $M$ be a matroid on a ground set $S$ that is not representable over $GF(2)$.
We want to show that $M$ has a $U_{2, 4}$ minor.
In order to do so, we will find a representable matroid $N$ and prove that one of $N$ and $M$ has to have a $U_{2, 4}$ minor.
First, without loss of generality, assume that $S = \{1, 2, 3, \cdots, n\}$ for some positive integer $n$.
Let $B$ be a basis of $M$.
Without loss of generality, $B = \{1, 2, 3, \cdots, k\}$.
Let $v_i$ be a column vector for each $i = 1, 2, \cdots, k$ with $k$ elements such that $i$th element is 1 and other elements are 0.
For each $j = k + 1, k + 2, \cdots, n$, let $C_j$ be the unique circuit in $B \cup \{ j \}$. 
(The existence is guaranteed by the lemma above).
Let $\displaystyle v_j = \sum_{i \in C - j } v_i$.
Now, we have defined $v_i$ for $i = 1, \cdots, n$.
Let $N$ be a column matroid of $\begin{pmatrix} v_1 v_2 \cdots v_n \end{pmatrix}$.
Obviously $B$ is a basis of $N$ from the way we defined $N$.
Now, we want to prove this property: For each $b \in B, s \in S \setminus B, B - b + s$ is a basis of $M$ if and only if it is a base of $N$.
Suppose $B - b + s$ is dependent in $M$.
Since $B - b + s$ is a dependent subset of $B + s$, $B - b + s$ must contain a unique circuit of $B + s$.
It means that $B - b$ contain indexes for vectors that can make $v_s$ dependent.
Therefore, $B - b + s$ is dependent in $N$.

Suppose $B - b + s$ is dependent in $N$.
It means that $B - b + s$ contains $C_s$ because $B$ is independent and because of the way $v_s$ is defined.
Therefore, $B - b + s$ is dependent in $M$.
% First, from the lemma above, we know that $B \cup \{ s \}$ has a unique circuit in $N$.
% The same goes for $M$.
% Moreover, they must be the same circuit from the way we defined $v_i$'s.
% 
% If $B - b + s$ is dependent in $M$, it contains a circuit $C$ of $M$.
% Since $C$ must be included in $B + s$, we know that $C$ is the unique circuit of $B + s$ in $M$.
% Since $C$ is also a circuit in $N$, $B - b + s$ is dependent in $N$.
% 
% On the other hand, suppose that $B - b + s$ is dependent in $N$.
% Let $C$ be a circuit of $N$ in $B - b + s$.
% Since $C$ must be in $B + s$, we know that $C$ is the unique circuit of $B + s$ in $N$.
% Since $C$ is also a circuit in $M$, $B - b + s$ is dependent in $M$.
% 
Since $B - b + s$ is independent in $N$ if and only if it is independent in $M$, we have proved the property.

Now, let $X$ such that $\lvert B \Delta X \rvert = 2$.
It means that there exists $b \in B, s \in S \setminus B$ such that $X = B - b + s$.
Therefore, $X$ is a basis of $M$ if and only if it is a basis of $N$.
In other words, $X$ is a basis of both $M$ and $N$ or neither of them.
Therefore, there is no $X$ such that $\lvert B \Delta X \rvert = 2$ and $X$ is a basis of exactly one of $M$ and $N$.
By lemma, we know that $M$ or $N$ has a $U_{2, 4}$ minor.
Clearly, it must be $M$ that has a $U_{2, 4}$ minor.
\end{proof}

However, finding a finite set of excluded minors for some finite case does not solve Rota's conjecture as it is more general.
In 2013, Geelen, Gerards and Whittle announced that they have solved Rota's Conjecture.
As they mention in \cite{solving} that ``there is a significant difference between the concrete problem of finding the full set of obstructions for some particular field and the abstract problem of showing that there are finitely many obstructions for an arbitrary finite field",
the approach they took was unique in a way that they started out by extending theorems from another field of mathematics to matroids.
Here is a brief summary of their proof.

As matroid theory is, in a way, an abstraction of graph theory, they have used several theorems from graph theory.
% Particularly, they extended Graph Minors Project to matroids.
% In \cite{solving}, they mention that this extension was really important since ``the Graph Minors Theory had exactly the kind of general purpose tools that we lacked".
For example, they extended the graph minor theorem to matroids.
Note that the minor of a graph can be obtained by deleting edges and vertices and by contracting edges.

\begin{thm}[Graph Minor Theorem]
Let $F$ be a minor-closed family of graphs, that is, $\forall G \in F$, any minor of $G$ is in $F$.
Then there are only finitely many graphs $H$ such that $H \notin F$ and any proper minor of $H$ is in $F$.
In other words, each minor-closed class of graphs has only finitely many excluded minors.
\end{thm}

One special case of this theorem is a set of planar graphs.
Any minor of a planar graph is also planar.
Therefore, a set $S$ of planar graphs is minor-closed.
Kuratowski's theorem indeed states that there are only two excluded minors, $K_{3, 3}, K_5$.

Here is the extension of the graph minor theorem to matroids.

\begin{thm} [Matroid WQO Theorem]
Let a finite field $\mathbb{F}$ be given and $F$ be a minor-closed family of $\mathbb{F}$-representable matroids.
Then there are only finitely many $\mathbb{F}$-representable matroids $M$ such that $M \notin F$ and any minor of $M$ is in $F$.
In other words, for each finite field $\mathbb{F}$ and each minor-closed class of $\mathbb{F}$-representable matroids, there are only finitely many $\mathbb{F}$-representable excluded minors.
\end{thm}

This theorem is theorem 6 from \cite{solving}.

More precisely, this is an extension from graphs to $\mathbb{F}$-representable matroids.

Although the Matroid WQO Theorem and the graph minor theorem both discuss excluded minors, the Matroid WQO Theorem does not imply Rota's Conjecture.
Rota's Conjecture is about a family of representable matroids, so all the excluded minors are non-representable.
In other words, obviously, there are never any $\mathbb{F}$-representable excluded minors.

%They mention that ``the most significant ingredient is an analogue of the Graph Minors Structure Theorem".
%
%(Put some description once I understand what Graph Minors Structure Theorem is)

Another important concept is connectivity.
\begin{defn}
A $k$-separation in a matroid $M = (E, \mathcal{I})$ is a partition $(X, Y)$ of $E$ such that $r_M(X) + r_M(Y) - r_M(E) < k$ and $\lvert X \rvert, \lvert Y \rvert \geq k$.
\end{defn}
\begin{defn}
A matroid is $k$-connected if it has no $l$-separation for any $l < k$.
\end{defn}

\begin{lem}
For each field $\mathbb{F}$, each excluded minor for the class of $\mathbb{F}$-representable matroids is 3-connected.
\end{lem}

This lemma is from \cite{solving}.

There are a few basic results about connectivity.

\begin{thm}
No matroid has 0-separation.
\end{thm}
\begin{proof}
Let $I$ be a maximal independent set.
Then $\lvert I \rvert = r_M(E)$.
Let $I_X = I \cap X, I_Y = I \cap Y$.
Then $\lvert I_X \rvert \leq r_M(X), \lvert I_Y \rvert \leq r_M(Y)$.
Therefore, $r_M(X) + r_M(Y) - r_M(E) \geq 0$
\end{proof}


\begin{thm}
$U_{2, 4}$ is indeed 3-connected.
\end{thm}
$U_{2, 4}$ is an excluded minor for $GF(2)$ as mentioned earlier, so this theorem should be true.
\begin{proof}
Let a partition of $E$, $(X, Y)$, be given.
Without loss of generality, $\lvert X \rvert \leq \lvert Y \rvert$.
Consider the case when $\lvert X \rvert = 1$.
Then we know that $r_M(X) = 1, r_M(Y) = 2$.
Consider the case when $\lvert X \rvert = 2$.
Then we know that $r_M(X) = 2, r_M(Y) = 2$.
Since $r_M(X) + r_M(Y) - r_M(E) \geq 1$ for any $(X, Y)$, a 1-partition does not exist.
Since $r_M(X) + r_M(Y) - r_M(E) < 2$ only if $r_M(X) = 1$, a 2-partition does not exist either.
Therefore, $U_{2, 4}$ is indeed 3-connected.
\end{proof}

The connectivity is important because it turns out that excluded minors are very highly connected in some sense, and that is one of the important parts of the rest of the proof, and the outline can be found in \cite{solving}.

% A weak version of this lemma can be proved as following:
% 
% \begin{lem}
% Let a finite field $\mathbb{F}$ and a matroid $M$ be given.
% If $M$ has a 1-separation $(X, Y)$, then $M$ is not an excluded minor.
% In other words, either it is representable or some of its minor is unrepresentable.
% \end{lem}
% 
% \begin{proof}
% $r_M(X) + r_M(Y) \geq r_M(E)$ since a maximal independent set must be split into $X, Y$.
% Therefore, the existence of 1-separation implies that $r_M(X) + r_M(Y) = r_M(E)$.
% Without loss of generality, $\lvert X \rvert \leq \lvert Y \rvert$.
% Suppose $\lvert X \rvert \geq 2$.
% I think I can prove this, but I haven't thought this through...
% \end{proof}
% 


% Chapter 5
% bibliography

\begin{thebibliography}{10}

\bibitem{amsshort}
"WHAT IS A MATROID?" Oxley
\bibitem{amsshort}
"On Matroid Representability and Minor Problems" Hlineny
\bibitem{amsshort}
"Solving Rota’s Conjecture" Geelen, Gerards, Whittle

\end{thebibliography}


\end{document}

