\section{Introduction to representability}

This chapter will introduce definitions and results that are necessary to discuss the representability.

More specifically, this chapter will introduce:
\begin{itemize}
\item Why we care about representability
  \begin{itemize}
  \item If all matroids are representable, we are just giving a vector space another name.
  \item Therefore, it is important that not all matroids are representable.
  \item Show some examples of unrepresentable matroids
  \item Now that we know the existence, the question is, which one?
  \end{itemize}
\item The mathematical definition of "representability"
\end{itemize}


\begin{defn}
Let $M = (E, \mathcal{I})$, $X \subseteq E$ be given.
$M \backslash X$ denotes a deletion of $X$ in $M$ and is defined to be $(E - X, \{ I \in \mathcal{I} \mid X \cap I = \emptyset \})$
\end{defn}

Note that $M \backslash e$ for some element $e \in E$ is equivalent to $M \backslash \{ e \}$.

\begin{thm}
Let $M = (E, \mathcal{I})$, $X \subseteq E$ be given.
$M \backslash X$ is indeed a matroid
\end{thm}

\begin{proof}
Let $M' = (E', \mathcal{I}') = M \backslash X$.
Since $\emptyset \in \mathcal{I}$ and $X \cap \emptyset = \emptyset$, $\emptyset \in \mathcal{I}$.
Let $I \in \mathcal{I}', J \subseteq I$. 
Since $I \in \mathcal{I}$, $J \in \mathcal{I}$.
Since $J \subseteq I$ and $I \cap X = \emptyset$, $J \cap X = \emptyset$.
Therefore, $J \in \mathcal{I}'$.
Let $A, B \in \mathcal{I}'$ such that $\lvert A \rvert < \lvert B \rvert$.
Since $A, B \in \mathcal{I}'$, we know that $A, B \in \mathcal{I}$.
Therefore, we can find $x \in B - A$ such that $(A \cup \{ x \}) \in \mathcal{I}$.
Since $(A \cup \{ x \}) \subseteq (A \cup B)$ and $X \cap A = X \cap B = \emptyset$, $X \cap (A \cup \{ x \}) = \emptyset$.
Therefore, $A \cup \{ x \} \in \mathcal{I}'$.
Thus we have found such $x \in B - A$ that $A \cup \{ x \} \in \mathcal{I}'$.
Since $M' = M \backslash X$ follows the three properties, it is indeed a matroid. 
\end{proof}

\begin{defn}
Let $M = (E, \mathcal{I})$, $e \in E$ be given.
$M / e$ denotes contraction of $M$ by $e$ and 
$M / e = \begin{cases}
      M \backslash e, \text{if $e$ is a loop},\\
      (E - \{ e \}, \{ I \in \mathcal{I} \mid e \notin I, (I \cup \{ e \}) \in \mathcal{I}\}), \text{ otherwise}.
         \end{cases}$
\end{defn}
\begin{thm}
Contraction by an element indeed generates a matroid.
\end{thm}
\begin{proof}
If $e$ is a loop, $M / e$ is obviously a matrod since we know that deletion always generates a matroid.
Suppose otherwise.
Let $\mathcal{I}'$ denote the independent sets of $M / e$.
First, $\emptyset \in \mathcal{I}, e \notin \emptyset$. Since $e$ is not a loop, $(\emptyset \cup \{ e \}) \in \mathcal{I}$.
Therefore, $\emptyset \in \mathcal{I}'$.
Let $I \in \mathcal{I}', J \subseteq I$.
Since $I \in \mathcal{I}$, $J \in \mathcal{I}$.
Since $e \notin I$, $e \notin J$.
Since $(I \cup \{ e \}) \in \mathcal{I}$ and $J \subseteq I$, $(J \cup \{ e \}) \in \mathcal{I}$.
Therefore, $J \in \mathcal{I}'$.
** Read this part more carefully.**
Let $A, B \in \mathcal{I}'$ such that $\lvert A \rvert < \lvert B \rvert$.
Let $A' = A \cup \{ e \}, B' = B \cup \{ e \}$.
Since $A, B \in \mathcal{I}'$, $A', B' \in \mathcal{I}$.
Since $e \notin A, e \notin B$, $\lvert A' \rvert < \lvert B' \rvert$.
Let $x \in B' - A'$ such that $A' \cup \{ x \} \in \mathcal{I}$.
For such $x$, we just showed that $A \cup \{ e \} \cup \{ x \} \in \mathcal{I}$.
Also, $x \neq e$ since $e \in A'$.
Therefore, $A \cup \{ x \} \in \mathcal{I}'$. 
Hence, we have found $x \in B - A$ such that $A \cup \{ x \} \in \mathcal{I}'$.
Since this follows three properties given in the definition, this is indeed a matroid.
\end{proof}


Now that we have defined contraction of a matroid by an element, we can define contraction by a subset of a ground set.
\begin{defn}
Let $M = (E, \mathcal{I}), X = \{ x_1, \cdots, x_k \} \subseteq E$.
$M / X$ is defined to be $(((M/x_1)/x_2) \cdots)/x_k)$.
\end{defn}

It is not obvious that this is well-defined. 
In other words, it is not obvious that the order of contraction does not matter.
The following theorem shows that the order does not matter.

\begin{thm}
For any given matroid $M = (E, \mathcal{I})$,
$(M / e_1) / e_2 = (M / e_2) / e_1$ for any $e_1 \neq e_2 \in E$.
\end{thm}

\begin{proof}
Prove it!
\end{proof}

\subsection{What do contraction and deletion mean in graphs and vector spaces?}

\subsection{Why do these matter?}
Because if a matroid is representable over some field $$\mathbb{F}$$, its minor is always representable over $$\mathbb{F}$$.

\subsection{The introduction of Rota's conjecture}

