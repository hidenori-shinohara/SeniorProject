\section{Introduction to representability}

%This chapter will introduce definitions and results that are necessary to discuss the representability.
%
%More specifically, this chapter will introduce:
%\begin{itemize}
%\item Why we care about representability
%  \begin{itemize}
%  \item If all matroids are representable, we are just giving a vector space another name.
%  \item Therefore, it is important that not all matroids are representable.
%  \item Show some examples of unrepresentable matroids
%  \item Now that we know the existence, the question is, which one?
%  \end{itemize}
%\item The mathematical definition of "representability"
%\end{itemize}

We will start this chapter by defining the representability of a matroid.

\begin{defn}
A matroid $M = (E, \mathcal{I})$ is representable over a field $\mathbb{F}$ if 
there exists a matrix $A$ over $\mathbb{F}$ such that the column matroid of $A$ is isomorphic to $M$.
\end{defn}

Therefore, if the matroid indeed succeeded in abstracting the concept of linearly independence, 
there should be some matroids that are \textit{not} representable over some fields.
If all matroids are representable over every field, it means that we are simply discussing linear algebra using different terms.

Before introducing some unrepresentable matroids, we will start by introducing a nice property of representable matroids.

\begin{thm}
Let $M = (E, \mathcal{I})$ be a matroid that is representable over $\mathbb{F}$.
Let $r$ be a rank of $M$ and $k = \lvert E \rvert$.
Then there exists a matrix $A \in \mathbb{F}^{r \times k}$ such that $M$ is isomorphic to the column matroid of $A$.
\end{thm}

\begin{proof}
Let $B$ a matrix over $\mathbb{F}$ such that the column matroid of $B$ is isomorphic to $M$.
It is easy to see that the number of columns of $B$ is $k$.
From linear algebra, we know that elementary row operations preserve the linearly independency of column vectors.
Let $R$ be the reduced row echelon form of $B$.
Since $R$ must have a rank of $r$, it only has $r$ leading zeros.
In other words, $R$ has exactly $r$ non-zero row vectors.
Removing zero rows clearly does not affect the linearly independency.
Therefore, we found a matrix in $\mathbb{F}^{r \times k}$ whose column matroid is isomorphic to $M$.
\end{proof}

This property is useful when proving that a matroid is unrepresentable over some field.

Here are some matroids that are not representable over some fields to show that not all matroids are representable over every field.

\begin{thm}
$U_{2, 4}$ is not representable over $GF(2)$.
\end{thm}

A matroid is called a \textit{binary matroid} if it is representable over $GF(2)$.

\begin{proof}
We know that $U_{2, 4}$ has a rank of $2$.
If $U_{2, 4}$ is representable over $GF(2)$, there exists a matrix $A \in GF(2)^{2 \times 4}$ such that $A$'s column matroid is isomorphic to $U_{2, 4}$.
Since $GF(2)$ only has two elements, there are only four possible column vectors of size 2.
$\bigg\{\begin{pmatrix}0\\0\end{pmatrix},
\begin{pmatrix}0\\1\end{pmatrix},
\begin{pmatrix}1\\0\end{pmatrix},
\begin{pmatrix}1\\1\end{pmatrix}\bigg\}$
$A$ must not contain $\begin{pmatrix}0\\0\end{pmatrix}$ since it is a loop. 
Since $A$ has 4 columns and there are only 3 different column vectors, we know that there are two columns in $A$ that have the exact same column vectors.
This is also a contradiction since a subset of such two elements will not be independent.
Therefore, $U_{2, 4}$ is not representable over $GF(2)$.
\end{proof}

$U_{2, 4}$ is representable over some field such as $\mathbb{R}$.
For example, the column matroid of $\begin{pmatrix}
1 & 1 & 1 & 1 \\
1 & 2 & 3 & 4 
\end{pmatrix}$ is isomorphic to $U_{2, 4}$.


Now, we will introduce Fano matroid.
Fano matroid can be constructed from Fano plane.
Fano matroid is one of the examples of matroids that are representable over $GF(2)$, but not over $\mathbb{R}$.
We will start by defining Fano matroid mathematically.

\begin{defn}
Fano matroid.  A set of vertices is independent if it has at most three points and is not a line.
\end{defn}

\begin{thm}
Fano matroid is indeed a matroid
\end{thm}

\begin{proof}
Prove it!
\end{proof}


Here is an interesting property of Fano matroid
\begin{thm}
If Fano matroid is representable over a field $F$, $1 + 1 = 0$ in that field.
\end{thm}

\begin{proof}
meh
\end{proof}

This theorem is powerful, since this implies that Fano matroid is not representable over $\mathbb{R}$.
\begin{cor}
Fano matroid is not representable over $\mathbb{R}$.
\end{cor}

\begin{proof}
In $\mathbb{R}$, $1 + 1 \neq 0$. Therefore, Fano matroid is not representable over $\mathbb{R}$.
\end{proof}


\begin{thm}
If a matroid $M = (E, \mathcal{I})$ only contains at most 3 non-loop elements, it is representable over any field $\mathbb{F}$.
\end{thm}

A matroid is called \textit{regular} if it can be represented over any field.

\begin{proof}
I'll use rank here.
\end{proof}


