\section{Introduction to representability}

This chapter will introduce definitions and results that are necessary to discuss the representability.

More specifically, this chapter will introduce:
\begin{itemize}
\item Why we care about representability
  \begin{itemize}
  \item If all matroids are representable, we are just giving a vector space another name.
  \item Therefore, it is important that not all matroids are representable.
  \item Show some examples of unrepresentable matroids
  \item Now that we know the existence, the question is, which one?
  \end{itemize}
\item The mathematical definition of "representability"
\end{itemize}


\begin{defn}
Let $M = (E, \mathcal{I})$, $X \subseteq E$ be given.
$M \backslash X$ denotes a deletion of $X$ in $M$ and is defined to be $(E - X, \{ I \in \mathcal{I} \mid X \cap I = \emptyset \})$
\end{defn}

Note that $M \backslash e$ for some element $e \in E$ is equivalent to $M \backslash \{ e \}$.

\begin{thm}
Let $M = (E, \mathcal{I})$, $X \subseteq E$ be given.
$M \backslash X$ is indeed a matroid
\end{thm}

\begin{proof}
Let $M' = (E', \mathcal{I}') = M \backslash X$.
Since $\emptyset \in \mathcal{I}$ and $X \cap \emptyset = \emptyset$, $\emptyset \in \mathcal{I}$.
Let $I \in \mathcal{I}', J \subseteq I$. 
Since $I \in \mathcal{I}$, $J \in \mathcal{I}$.
Since $J \subseteq I$ and $I \cap X = \emptyset$, $J \cap X = \emptyset$.
Therefore, $J \in \mathcal{I}'$.
Let $A, B \in \mathcal{I}'$ such that $\lvert A \rvert < \lvert B \rvert$.
Since $A, B \in \mathcal{I}'$, we know that $A, B \in \mathcal{I}$.
Therefore, we can find $x \in B - A$ such that $(A \cup \{ x \}) \in \mathcal{I}$.
Since $(A \cup \{ x \}) \subseteq (A \cup B)$ and $X \cap A = X \cap B = \emptyset$, $X \cap (A \cup \{ x \}) = \emptyset$.
Therefore, $A \cup \{ x \} \in \mathcal{I}'$.
Thus we have found such $x \in B - A$ that $A \cup \{ x \} \in \mathcal{I}'$.
Since $M' = M \backslash X$ follows the three properties, it is indeed a matroid. 
\end{proof}

\begin{defn}
Let $M = (E, \mathcal{I})$, $e \in E$ be given.
$M / e$ denotes contraction of $M$ by $e$ and 
$M / e = \begin{cases}
      M \backslash e, \text{if $e$ is a loop},\\
      (E - \{ e \}, \{ I \in \mathcal{I} \mid e \notin I, (I \cup \{ e \}) \in \mathcal{I}\}), \text{ otherwise}.
         \end{cases}$
\end{defn}


\begin{thm}
Contraction by an element indeed generates a matroid.
\end{thm}

\begin{proof}
If $e$ is a loop, $M / e$ is obviously a matrod since we know that deletion always generates a matroid.
Suppose otherwise.
Let $\mathcal{I}'$ denote the independent sets of $M / e$.
First, $\emptyset \in \mathcal{I}, e \notin \emptyset$. Since $e$ is not a loop, $(\emptyset \cup \{ e \}) \in \mathcal{I}$.
Therefore, $\emptyset \in \mathcal{I}'$.
Let $I \in \mathcal{I}', J \subseteq I$.
Since $I \in \mathcal{I}$, $J \in \mathcal{I}$.
Since $e \notin I$, $e \notin J$.
Since $(I \cup \{ e \}) \in \mathcal{I}$ and $J \subseteq I$, $(J \cup \{ e \}) \in \mathcal{I}$.
Therefore, $J \in \mathcal{I}'$.
Let $A, B \in \mathcal{I}'$ such that $\lvert A \rvert < \lvert B \rvert$.
Let $A' = A \cup \{ e \}, B' = B \cup \{ e \}$.
Since $A, B \in \mathcal{I}'$, $A', B' \in \mathcal{I}$.
Since $e \notin A, e \notin B$, $\lvert A' \rvert < \lvert B' \rvert$.
Let $x \in B' - A'$ such that $A' \cup \{ x \} \in \mathcal{I}$.
Since $B' - A' = B - A$, $x \in B - A$.
For such $x$, we just showed that $A \cup \{ e \} \cup \{ x \} \in \mathcal{I}$.
Also, $x \neq e$ since $e \in A'$.
Therefore, $A \cup \{ x \} \in \mathcal{I}'$. 
Hence, we have found $x \in B - A$ such that $A \cup \{ x \} \in \mathcal{I}'$.
Since this follows three properties given in the definition, this is indeed a matroid.
Therefore, contraction by an element indeed generates a matroid.
\end{proof}


Here are few simple yet useful results about deletion.

\begin{thm}
Let a matroid $M = (E, \mathcal{I}), X \subseteq E$ be given.
Let $A$ be an independent set in $M \backslash X$.
Then $A$ is independent in $M$.
\end{thm}

\begin{proof}
The independent sets of $M \backslash X$ is $\{ I \in \mathcal{I} \mid (I \cap X) = \emptyset \}$.
It is easy to see that it is a subset of $\mathcal{I}$.
Since $A$ is in the subset of $\mathcal{I}$, $A$ must be in $\mathcal{I}$.
\end{proof}

In other words, this means that deletion never "adds" a new element to the independent sets.


\begin{thm}
The deletion of a loop does not change the independent sets.
\end{thm}

\begin{proof}
Let $M = (E, \mathcal{I}), e \in E, \{ e \} \notin \mathcal{I}$.
The independent sets of $M \backslash e$ is $\{ I \in \mathcal{I} \mid (I \cap \{ e \}) = \emptyset \}$.
Since $\{ e \}$ is a loop, no independent set can contain $e$.
Therefore, the independent sets of $M \backslash e$ is identical to $\mathcal{I}$.
\end{proof}

Now that we have defined contraction of a matroid by an element, we can define contraction by a subset of a ground set.
\begin{defn}
Let $M = (E, \mathcal{I}), X = \{ x_1, \cdots, x_k \} \subseteq E$.
$M / X$ is defined to be $(((M/x_1)/x_2) \cdots)/x_k)$.
\end{defn}

It is not obvious that this is well-defined. 
In other words, it is not obvious that the order of contraction does not matter.
The following theorem shows that the order does not matter.

\begin{thm}
For any given matroid $M = (E, \mathcal{I})$,
$(M / e) / f = (M / f) / e$ for any $e \neq f \in E$.
\end{thm}

\begin{proof}
There are a few cases.
\begin{enumerate}

\item $e, f$ are both loops. \\
$(M / e) / f = (M \backslash e) / f$
Since deletion of a loop does not change the independent sets, $f$ is a loop in $(M \backslash e)$.
Therefore, $(M / e) / f = (M \backslash e) \backslash f$.
Again, deletion of $f$ does not change the independent sets since $f$ is a loop in $(M / e)$.
Therefore, we have $(M / e) / f = (E - \{ e, f \}, \mathcal{I})$.
By symmetry, $(M / e) / f = (M / f) / e$.

\item 
One of $e, f$ is a loop, and the other one is not. \\
Without loss of generality, assume $e$ is a loop.
$(M / e) / f = (M \backslash e) / f$.
Since deletion of a loop does not change the independent set, the independent set of $(M / e)$ is $\mathcal{I}$.
Therefore, the independent set of $(M / e) / f$ is $\mathcal{I}' = \{ I \in \mathcal{I} \mid f \notin I, (I \cup \{ f \}) \in \mathcal{I} \}$.
On the other hand, it is easy to see that $\mathcal{I}'$ is identical to the independent sets of $(M / f)$.
Since contraction by an element does not add new elements to the independent sets, $e$ is a loop in $(M / f)$.
Since deletion by a loop does not change the independent sets, the independent sets of $(M / f) / e$ is $\mathcal{I}'$.
Now we confirmed that $(M / e) / f$ and $(M / f) / e$ have the same independent sets.
Therefore, $(M / e) / f = (M / f) / e$.

\item Neither of them is a loop, and $\{ e, f \} \in \mathcal{I}$. \\
The independent set of $M / e$ is $\mathcal{I}' = \{ I \in \mathcal{I} \mid e \notin I, (I \cup \{ e \}) \in \mathcal{I} \}$.
Since $\{ e, f \} \in \mathcal{I}$, $\{ f \} \in \mathcal{I}'$. 
Therefore, $f$ is not a loop in $M / e$.
Hence, the independent sets of $(M / e) / f$ is $\mathcal{I}'' = \{ I \in \mathcal{I}' \mid f \notin I, (I \cup \{ f \}) \in \mathcal{I}' \}$.
$\mathcal{I}''$ is actually equivalent to $S = \{ I \in \mathcal{I} \mid e \notin I, f \notin I, (I \cup \{ e, f \}) \in \mathcal{I} \}$.
We can prove $\mathcal{I}'' = S$ by starting to show that $\mathcal{I}'' \subseteq S$.
Let $I \in \mathcal{I}''$.
Since $I$ is an independent set of $(M / e) / f$, we know that $e, f \notin I$.
Since $(I \cup \{ f \}) \in \mathcal{I}'$, we also know that $((I \cup \{ f \}) \cup \{ e \}) \in \mathcal{I}$.
Therefore, $(I \cup \{ e, f \}) \in \mathcal{I}$.
Thus $I \in S$, and $\mathcal{I}'' \subseteq S$.
Now, we want to show that $S \subseteq \mathcal{I}''$. 
Let $I \in S$. By the definition of $S$, we know that $e, f \notin I$.
Since $(I \cup \{ e, f \}) \in \mathcal{I}$, we know that $(I \cup \{ e \}) \in \mathcal{I}$.
Since $((I \cup \{ f \}) \cup \{ e \}) \in \mathcal{I}$ and $e \notin (I \cup \{ f \})$, we know that $(I \cup \{ f \}) \in \mathcal{I}'$.
Since $f \notin I$ and $(I \cup \{ f \}) \in \mathcal{I}'$, $I \in \mathcal{I}''$.
Hence, $S \subseteq \mathcal{I}''$.

Combining these two results, we know that $S = \mathcal{I}''$.
By the symmetry, $(M / e) / f$ and $(M / f) / e$ have the same independent sets.
Therefore $(M / e) / f = (M / f) / e$.
\end{enumerate}

\end{proof}



\subsection{What do contraction and deletion mean in graphs and vector spaces?}

\subsection{Why do these matter?}
Because if a matroid is representable over some field $$\mathbb{F}$$, its minor is always representable over $$\mathbb{F}$$.

\subsection{The introduction of Rota's conjecture}

