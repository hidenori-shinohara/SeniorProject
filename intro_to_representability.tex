\section{Introduction to representability}

%This chapter will introduce definitions and results that are necessary to discuss the representability.
%
%More specifically, this chapter will introduce:
%\begin{itemize}
%\item Why we care about representability
%  \begin{itemize}
%  \item If all matroids are representable, we are just giving a vector space another name.
%  \item Therefore, it is important that not all matroids are representable.
%  \item Show some examples of unrepresentable matroids
%  \item Now that we know the existence, the question is, which one?
%  \end{itemize}
%\item The mathematical definition of "representability"
%\end{itemize}

We will start this chapter by defining the representability of a matroid.

\begin{defn}
A matroid $M = (E, \mathcal{I})$ is representable over a field $\mathbb{F}$ if 
there exists a matrix $A$ over $\mathbb{F}$ such that the column matroid of $A$ is isomorphic to $M$.
\end{defn}

Therefore, if the matroid indeed succeeded in abstracting the concept of linear independence, 
there should be some matroids that are \textit{not} representable over some fields.
If all matroids are representable over every field, it means that we are simply discussing linear algebra using different terms.

Before introducing some unrepresentable matroids, we will start by introducing a nice property of representable matroids.

\begin{thm}
Let $M = (E, \mathcal{I})$ be a matroid that is representable over $\mathbb{F}$.
Let $r$ be a rank of $M$ and $k = \lvert E \rvert$.
Then there exists a matrix $A \in \mathbb{F}^{r \times k}$ such that $M$ is isomorphic to the column matroid of $A$.
\end{thm}

\begin{proof}
Let $B$ a matrix over $\mathbb{F}$ such that the column matroid of $B$ is isomorphic to $M$.
It is easy to see that the number of columns of $B$ is $k$.
From linear algebra, we know that elementary row operations preserve the linear independence of column vectors.
Let $R$ be the reduced row echelon form of $B$.
Since $R$ must have a rank of $r$, it only has $r$ leading zeros.
In other words, $R$ has exactly $r$ non-zero row vectors.
Removing zero rows clearly does not affect the linear independence.
Therefore, we found a matrix in $\mathbb{F}^{r \times k}$ whose column matroid is isomorphic to $M$.
\end{proof}

This property is useful when proving that a matroid is unrepresentable over some field.

Here are some matroids that are not representable over some fields to show that not all matroids are representable over every field.

\begin{thm}
$U_{2, 4}$ is not representable over $GF(2)$.
\end{thm}

A matroid is called a \textit{binary matroid} if it is representable over $GF(2)$.

\begin{proof}
We know that $U_{2, 4}$ has a rank of $2$.
If $U_{2, 4}$ is representable over $GF(2)$, there exists a matrix $A \in GF(2)^{2 \times 4}$ such that $A$'s column matroid is isomorphic to $U_{2, 4}$.
Since $GF(2)$ only has two elements, there are only four possible column vectors of size 2.
$\bigg\{\begin{pmatrix}0\\0\end{pmatrix},
\begin{pmatrix}0\\1\end{pmatrix},
\begin{pmatrix}1\\0\end{pmatrix},
\begin{pmatrix}1\\1\end{pmatrix}\bigg\}$
$A$ must not contain $\begin{pmatrix}0\\0\end{pmatrix}$ since it is a loop. 
Since $A$ has 4 columns and there are only 3 different column vectors, we know that there are two columns in $A$ that have the exact same column vectors.
This is a contradiction since a subset of such two elements will not be independent.
Therefore, $U_{2, 4}$ is not representable over $GF(2)$.
\end{proof}

$U_{2, 4}$ is representable over some field such as $\mathbb{R}$.
For example, the column matroid of $\begin{pmatrix}
1 & 1 & 1 & 1 \\
1 & 2 & 3 & 4 
\end{pmatrix}$ is isomorphic to $U_{2, 4}$.


Now, we will introduce Fano matroid.
Fano matroid can be constructed from Fano plane.
Fano matroid is one of the examples of matroids that are representable over $GF(2)$, but not over $\mathbb{R}$.
We will start by defining Fano matroid mathematically.

\begin{figure}
  \centering
    \includegraphics[width=0.8\textwidth,natwidth=610,natheight=642]{Fano_plane.png}
    \caption{Fano plane}
  \label{fig:test}
\end{figure}

\begin{defn}
Fano matroid is a matroid with a ground set $\{ 1, 2, \cdots, 7 \}$. 
A set of vertexes is independent if it satisfies one of the followings:
\begin{enumerate}
\item it contains less than or equal to 2 elements,
\item it contains exactly 3 points and they are not on the same line.
\end{enumerate}
Any set of vertexes that have more than 3 elements is dependent.
\end{defn}

For example, $\{4, 5\}$ and $\{ 1, 2, 3 \}$ are independent, but $\{ 1, 4, 2 \}$, $\{1, 5, 7\}$, and $\{4, 5, 6 \}$ are not independent as each of them is on one line.

\begin{thm}
Fano matroid is indeed a matroid.
\end{thm}

\begin{proof}
An empty set is independent since it contains less than 2 elements.
Any independent set has at most three elements, so any proper subset of it has at most two elements.
Therefore, any subset of independent sets is always independent.
The third property can be proved by checking each case.
Since any set with 2 or fewer elements is independent, we only need to care about the case when we have a set with 2 elements and a set of three elements.
Let $A = \{ a_1, a_2, a_3 \}$, $B = \{ b_1, b_2\}$ be independent sets.
It is easy to see from the figure that there must be exactly one line that goes through both $b_1$ and $b_2$.
Let $x$ denote the third point on such line.
Then, adding any element other than $b_1, b_2, x$ to $B$ will generate an independent set of three elements.
Therefore, we just need to make sure that $A \neq \{ b_1, b_2, x \}$.
That cannot be the case since $\{ b_1, b_2, x \}$ is dependent and $A$ is independent.
Therefore, there must be an element $a \in A - B$ such that $B \cup \{ a \}$ is independent.
Since Fano matroid satisfies all three properties, it is indeed a matroid.
\end{proof}


Here is an interesting property of Fano matroid
\begin{thm}
If Fano matroid is representable over a field $\mathbb{F}$, $1 + 1 = 0$ in that field.
\end{thm}

\begin{proof}
Suppose Fano matroid is representable over a given field $\mathbb{F}$.
Let $A \in \mathbb{F}^{3 \times 7}$ such that $A$'s column matroid is isomorphic to Fano matroid.
Let $R$ be a row reduced echelon form of $A$.
Then the first three columns should be identical to $I_3$ since $R$ has a rank of 3.
Since $\{1, 2, 4\}$ is dependent, $R_{3, 4}$ is 0.
Applying the same argument to $\{2, 3, 5\}, \{1, 3, 6\}$, we get the following:
$\begin{pmatrix}
1 & 0 & 0 & ? & 0 & ? & ? \\
0 & 1 & 0 & ? & ? & 0 & ? \\
0 & 0 & 1 & 0 & ? & ? & ? \\
\end{pmatrix}$
Since multiplying a non-zero constant to some column does not affect on the linearly independency, assume that the first non-zero elements of 4, 5, 6th columns are all 1.
$\begin{pmatrix}
1 & 0 & 0 & 1 & 0 & 1 & ? \\
0 & 1 & 0 & ? & 1 & 0 & ? \\
0 & 0 & 1 & 0 & ? & ? & ? \\
\end{pmatrix}$

Let $a = R_{2, 4}, b = R_{3, 5}$.
$\begin{pmatrix}
1 & 0 & 0 & 1 & 0 & 1 & ? \\
0 & 1 & 0 & a & 1 & 0 & ? \\
0 & 0 & 1 & 0 & b & ? & ? \\
\end{pmatrix}$
Since $\{4, 5, 6\}$ is dependent, $R_{3, 6}$ must be $-ab$.
$\begin{pmatrix}
1 & 0 & 0 & 1 & 0 & 1 & ? \\
0 & 1 & 0 & a & 1 & 0 & ? \\
0 & 0 & 1 & 0 & b & -ab & ? \\
\end{pmatrix}$
We need to look into the seventh column.
The seventh column actually cannot contain any zero.
For example, suppose $R_{1, 7} = 0$.
Then, $\{2, 3, 7 \}$ would be dependent.
That's a contradiction.
Similar arguments apply to the case of $R_{2, 7} = 0, R_{3, 7} = 0$.
By multiplying a non-zero constant, we get:
$\begin{pmatrix}
1 & 0 & 0 & 1 & 0 & 1 & 1 \\
0 & 1 & 0 & a & 1 & 0 & ? \\
0 & 0 & 1 & 0 & b & -ab & ? \\
\end{pmatrix}$
Since $\{3, 4, 7\}$ is dependent, $R_{2, 7} = a$.
And since $\{1, 5, 7\}$ is dependent, $R_{3, 7} = b R_{2, 7} = ab$.
$\begin{pmatrix}
1 & 0 & 0 & 1 & 0 & 1 & 1 \\
0 & 1 & 0 & a & 1 & 0 & a \\
0 & 0 & 1 & 0 & b & -ab & ab \\
\end{pmatrix}$
$\{ 2, 6, 7 \}$ is dependent as well.
By observing those three columns, it is easy to see that $-(-ab) + ab$ must be 0.
Since neither of $a, b$ is 0, $ab(1 + 1) = 0$ implies $1 + 1 = 0$.
(If $a$ is 0, 1st column and 4th column would be identical. The similar argument applies to $b$.)
\end{proof}

This theorem is powerful, since this implies that Fano matroid is not representable over $\mathbb{R}$.
\begin{cor}
Fano matroid is not representable over $\mathbb{R}$.
\end{cor}

\begin{proof}
In $\mathbb{R}$, $1 + 1 \neq 0$. Therefore, Fano matroid is not representable over $\mathbb{R}$.
\end{proof}


\begin{thm}
If a matroid $M = (E, \mathcal{I})$ only contains at most 3 non-loop elements, it is representable over any field $\mathbb{F}$.
\end{thm}

A matroid is called \textit{regular} if it can be represented over any field.

\begin{proof}
I'll use rank here.
\end{proof}

The number of matroids increases exponentially as the size of the ground set increases.

