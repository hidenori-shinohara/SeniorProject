\section{The introduction to Rota's Conjecture}
This chapter introduces Rota's Conjecture and an outline of the proof.

\begin{conj}[Rota's Conjecture]
For each finite field $\mathbb{F}$, there are, up to isomorphism, only finitely many excluded minors for the class of $\mathbb{F}$-representable matroids.
\end{conj}

In other words, given a finite field $\mathbb{F}$, let $F$ be a family of all matroids that are representable over $\mathbb{F}$.
By the theorem from the previous chapter, we know that $F$ is minor-closed, i.e., any minor of any element in $F$ is in $F$.
Then, an excluded minor is a matroid $M \notin F$ such that any proper minor of $M$ is in $F$.
The conjecture states that for each finite field $\mathbb{F}$, there are only finitely many excluded minors.

First, we will introduce some results related to excluded minors.

\begin{thm}
The Fano matroid is an excluded minor for the class of $\mathbb{F}$-representable matroids if the characteristic of $\mathbb{F}$ is not 2.
\end{thm}
\begin{proof}
Let $M$ be the Fano matroid.
We know from the previous chapter that $M$ is not representable in a field if the characteristic of the field is not 2.
Therefore, it suffices to show that any minor of $M$ is representable in those fields.
Since any minor of a representable matroid is always representable, it suffices to show that $M / e$ and $M \backslash e$ are representable for any $e$ in those fields.
It is obvious that $M / 1$, $M / 2$ and $M / 3$ are isomorphic to each other by the symmetry of the Fano plane.
For the same reason, each matroid in each of the following sets is isomorphic to other matroids in the same set.
$\{ M \backslash 1, M \backslash 2, M \backslash 3 \}, \{ M / 4, M / 5, M / 6 \}, \{ M \backslash 4, M \backslash 5, M \backslash 6 \}$.
If we swap 1 with 4, 3 with 5, the Fano place still gives the identical matroid.
That implies that deletion or contraction of any of 1, 2, 3, 4, 5, 6 always gives isomorphic matroids.
Similarly, if we swap 6 with 7, 3 with 5, the Fano place still gives the identical matroid.
That implies that deletion or contraction of any element always gives isomorphic matroids.
Now that we have shown the symmetry, it suffices to show that $M / 7$ and $M \backslash  7$ are both representable in those fields.
$M \backslash 7$ is isomorphic to the column matroid of the following matrix:
$\begin{pmatrix}
1 & 0 & 0 & 1 & 0 & 1 \\
0 & 1 & 0 & 1 & 1 & 0 \\
0 & 0 & 1 & 0 & 1 & -1 \\
\end{pmatrix}$, where -1 is the additive inverse of 1.
Similarly, $M / 7$ is isomorphic to the column matroid of the following matrix:
$\begin{pmatrix}
1 & 0 & 1 & 1 & 1 & 0 \\
0 & 1 & 1 & 1 & 0 & 1
\end{pmatrix}$.
\end{proof}

Rota's conjecture has been proved for some finite fields.
For example, $U_{2, 4}$ is the only excluded minor of $GF(2)$, and the proof can be found in \cite{lec9}.
In $GF(3)$, there are 4 excluded minors, which are $F_7, F_7^\ast, U_{2, 5}, U_{3, 5}$.\cite{lec9}
In $GF(4)$, there are 7 excluded minors, which are $U_{2, 6}, U_{4, 6}, P_6, F_7^-, (F_7^-)^\ast, P_8, P_8''$.\cite{gf4}

Here is a sketch of proof that $U_{2, 4}$ is the only excluded minor of $GF(2)$.
The complete proof can be found in \cite{lec9}.

First, we will assume the following lemmas. Proofs of them can be found in \cite{lec9}.

\begin{lem}
Let $B$ be a basis of a matroid $M = (S, \mathcal{I})$, and $e \notin B$.
Then $B \cup \{ e \}$ contains a unique circuit.
\end{lem}

\begin{lem}
Let $M, N$ be distinct matroids on the same ground set $S$.
Suppose that there exists $B \subseteq S$ such that
\begin{itemize}
\item $B$ is a basis of $M$ and $N$.
\item There is no $X$ such that $\lvert B \Delta X \rvert = 2$ and $X$ is a basis of exactly one of $M$ and $N$.
\end{itemize}
Then $M$ or $N$ has a $U_{2, 4}$ minor.
\end{lem}

\begin{thm}[Tutte's Theorem]
$U_{2, 4}$ is the only excluded minor of $GF(2)$.
In other words, a matroid $M$ is representable over $GF(2)$ if and only if $U_{2, 4}$ is not $M$'s minor.
\end{thm}

\begin{proof}
By theorem 2.4, we know that $U_{2, 4}$ is not representable over $GF(2)$.
Since any proper minor of $U_{2, 4}$ contains at most 3 elements, by theorem 2.9, we know that any proper minor of $U_{2, 4}$ is representable over $GF(2)$.
Therefore, it suffices to prove that any matroid $M$ that is not representable over $GF(2)$ has a $U_{2, 4}$ minor.
Let $M$ be such a matroid on a ground set $S$.
We will find a representable matroid $N$ and prove that one of $N$ and $M$ has to have a $U_{2, 4}$ minor.
First, without loss of generality, assume that $S = \{1, 2, 3, \cdots, n\}$ for some positive integer $n$.
Let $B$ be a basis of $M$.
Without loss of generality, $B = \{1, 2, 3, \cdots, k\}$.
Let $v_i$ be a column vector for each $i = 1, 2, \cdots, k$ with $k$ elements such that $i$th element is 1 and other elements are 0.
For each $j = k + 1, k + 2, \cdots n$, let $C$ be the unique circuit in $B \cup \{ j \}$. 
(The existence is guaranteed by the lemma above).
Let $\displaystyle v_j = \sum_{i \in C - s } v_i$.
Now, we have defined $v_i$ for $i = 1, \cdots, n$.
Let $N$ be a column matroid of $\begin{pmatrix} v_1 v_2 \cdots v_n \end{pmatrix}$.
Obviously $B$ is a basis of $N$ from the way we defined $N$.
Now, we want to prove this property: For each $b \in B, s \in S \setminus B, B - b + s$ is a basis of $M$ if and only if it is a base of $N$.
First, from the lemma above, we know that $B \cup \{ s \}$ has a unique circuit in $N$.
The same goes for $M$.
Moreover, they must be the same circuit from the way we defined $v_i$'s.

If $B - b + s$ is dependent in $M$, it contains a circuit $C$ of $M$.
Since $C$ must be included in $B + s$, we know that $C$ is the unique circuit of $B + s$ in $M$.
Since $C$ is also a circuit in $N$, $B - b + s$ is dependent in $N$.

On the other hand, suppose that $B - b + s$ is dependent in $N$.
Let $C$ be a circuit of $N$ in $B - b + s$.
Since $C$ must be in $B + s$, we know that $C$ is the unique circuit of $B + s$ in $N$.
Since $C$ is also a circuit in $M$, $B - b + s$ is dependent in $M$.

Since $B - b + s$ is independent in $N$ if and only if it is independent in $M$, we have proved the property.

Now, let $X$ such that $\lvert B \Delta X \rvert = 2$.
It means that there exists $b \in B, s \in S \setminus B$ such that $X = B - b + s$.
Therefore, $X$ is a basis of $M$ if and only if it is a basis of $N$.
In other words, $X$ is a basis of both $M$ and $N$ or neither of them.
Therefore, there is no $X$ such that $\lvert B \Delta X \rvert = 2$ and $X$ is a basis of exactly one of $M$ and $N$.
By lemma, we know that $M$ or $N$ has a $U_{2, 4}$ minor.
Clearly, it must be $M$ that has $U_{2, 4}$ minor.
\end{proof}

However, finding a finite set of excluded minors for some finite case does not solve Rota's conjecture as it is more general.
In 2013, Geelen, Gerards and Whittle announced that they have solved Rota's Conjecture.
As they mention in \cite{solving} that ``there is a significant difference between the concrete problem of finding the full set of obstructions for some particular field and the abstract problem of showing that there are finitely many obstructions for an arbitrary finite field",
the approach they took was unique in a way that they started out by extending theorems from another field of mathematics to matroids.
Here is a brief summary of their proof.

As matroid theory is, in a way, an abstraction of graph theory, they have used several theorems from graph theory.
% Particularly, they extended Graph Minors Project to matroids.
% In \cite{solving}, they mention that this extension was really important since ``the Graph Minors Theory had exactly the kind of general purpose tools that we lacked".
For example, they extended the graph minor theorem to matroids.
Note that the minor of a graph can be obtained by deleting edges and vertices and by contracting edges.

\begin{thm}[Graph Minor Theorem]
Let $F$ be a minor-closed family of graphs, that is, $\forall G \in F$, any minor of $G$ is in $F$.
Then there are only finitely many graphs $H$ such that $H \notin F$ and any minor of $H$ is in $F$.
In other words, each minor-closed class of graphs has only finitely many excluded minors.
\end{thm}

One special case of this theorem is a set of planar graphs.
Any minor of a planar graph is also planar.
Therefore, a set $S$ of planar graphs is minor-closed.
Kuratowski's theorem indeed states that there are only two excluded minors, $K_{3, 3}, K_5$.

Here is the extension of the graph minor theorem to matroids.

\begin{thm} [Matroid WQO Theorem]
Let a finite field $\mathbb{F}$ be given and $F$ be a minor-closed family of $\mathbb{F}$-representable matroids.
Then there are only finitely many $\mathbb{F}$-representable matroids $M$ such that $M \notin F$ and any minor of $M$ is in $F$.
In other words, for each finite field $\mathbb{F}$ and each minor-closed class of $\mathbb{F}$-representable matroids, there are only finitely many $\mathbb{F}$-representable excluded minors.
\end{thm}

This theorem is theorem 6 from \cite{solving}.

More precisely, this is an extension from graphs to $\mathbb{F}$-representable matroids.

Although the Matroid WQO Theorem and the graph minor theorem both discuss excluded minors, the Matroid WQO Theorem does not imply Rota's Conjecture.
Rota's Conjecture is about a family of representable matroids, so all the excluded minors are non-representable.
In other words, obviously, there are never any $\mathbb{F}$-representable excluded minors.

%They mention that ``the most significant ingredient is an analogue of the Graph Minors Structure Theorem".
%
%(Put some description once I understand what Graph Minors Structure Theorem is)

Another important concept is connectivity.
\begin{defn}
A $k$-separation in a matroid $M = (E, \mathcal{I})$ is a partition $(X, Y)$ of $E$ such that $r_M(X) + r_M(Y) - r_M(E) < k$ and $\lvert X \rvert, \lvert Y \rvert \geq k$.
\end{defn}
\begin{defn}
A matroid is $k$-connected if it has no $l$-separation for any $l < k$.
\end{defn}
% 
% \begin{lem}
% For each field $\mathbb{F}$, each excluded minor for the class of $\mathbb{F}$-representable matroids is 3-connected.
% \end{lem}
% 
% This lemma is from \cite{solving}.
% 

There are a few basic results about connectivity.

\begin{thm}
No matroid has 0-separation.
\end{thm}
\begin{proof}
Let $I$ be a maximal independent set.
Then $\lvert I \rvert = r_M(E)$.
Let $I_X = I \cap X, I_Y = I \cap Y$.
Then $\lvert I_X \rvert \leq r_M(X), \lvert I_Y \rvert \leq r_M(Y)$.
Therefore, $r_M(X) + r_M(Y) - r_M(E) \geq 0$
\end{proof}


\begin{thm}
$U_{2, 4}$ is indeed 3-connected.
\end{thm}
$U_{2, 4}$ is an excluded minor for $GF(2)$ as mentioned above, so this theorem should be true.
\begin{proof}
Let a partition of $E$, $(X, Y)$, be given.
Without loss of generality, $\lvert X \rvert \leq \lvert Y \rvert$.
Consider the case when $\lvert X \rvert = 1$.
Then we know that $r_M(X) = 1, r_M(Y) = 2$.
Consider the case when $\lvert X \rvert = 2$.
Then we know that $r_M(X) = 2, r_M(Y) = 2$.
Since $r_M(X) + r_M(Y) - r_M(E) \geq 1$ for any $(X, Y)$, a 1-partition does not exist.
Since $r_M(X) + r_M(Y) - r_M(E) < 2$ only if $r_M(X) = 1$, a 2-partition does not exist either.
Therefore, $U_{2, 4}$ is indeed 3-connected.
\end{proof}

The connectivity is important because it turns out that excluded minors are very highly connected, and that is one of the important parts of the rest of the proof, and the outline can be found in \cite{solving}.

% A weak version of this lemma can be proved as following:
% 
% \begin{lem}
% Let a finite field $\mathbb{F}$ and a matroid $M$ be given.
% If $M$ has a 1-separation $(X, Y)$, then $M$ is not an excluded minor.
% In other words, either it is representable or some of its minor is unrepresentable.
% \end{lem}
% 
% \begin{proof}
% $r_M(X) + r_M(Y) \geq r_M(E)$ since a maximal independent set must be split into $X, Y$.
% Therefore, the existence of 1-separation implies that $r_M(X) + r_M(Y) = r_M(E)$.
% Without loss of generality, $\lvert X \rvert \leq \lvert Y \rvert$.
% Suppose $\lvert X \rvert \geq 2$.
% I think I can prove this, but I haven't thought this through...
% \end{proof}
% 
