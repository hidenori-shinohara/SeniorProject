\section{The introduction of Rota's conjecture}

\begin{conj}
For each finite field $\mathbb{F}$, there are, up to isomorphism, only finitely many excluded minors for the class of $\mathbb{F}$-representable matroids.
\end{conj}(From \textit{Solving Rota's conjecture})


In other words, given a finite field $\mathbb{F}$, let $F$ be a family of all matroids that are representable over $\mathbb{F}$.
By the theorem from the previous chapter, we know that $F$ is minor-closed, i.e., any minor of any element in $F$ is in $F$.
Then an excluded minor is a matroid $M \notin F$ such that any minor of $M$ is in $F$.
The conjecture states that for any finite field $\mathbb{F}$, there are only finitely many excluded minors.

Geelen, Gerards, Whittle announced that they have solved Rota's Conjecture.
Here is a brief summary of their proof.

As matroid theory is partially an abstraction of graph theory, they have used several theorems from graph theory.
Here is one of the theorems from graph theory.
Note that the minor of a graph is a graph that can be obtained by contracting some edges, and it does not include deletion of edges.

\begin{thm}
Let $F$ be a minor-closed family of graphs, that is, $\forall G \in F$, any minor of $G$ is in $F$.
Then there are only finitely many graphs $H$ such that $H \notin F$ and any minor of $H$ is in $F$.
In other words, each minor-closed class of graphs has only finitely many excluded minors.
\end{thm}

This is called as \textit{Well-Quasi-Ordering Theorem}.
One special case of this theorem is a set of planar graphs.
Any minor of a planar graph is also planar.
Therefore, a set $S$ of planar graphs is minor-closed.
Kuratowski's theorem indeed states that there are only two excluded minors, $K_{3, 3}, K_5$.

This theorem is very similar to Rota's Conjecture as they are both about a minor-closed family of mathematical objects and claim that there are only finitely many excluded minors.

Here is a generalized version of WQO theorem.

\begin{thm}
Let a finite field $\mathbb{F}$ be given and $F$ be a minor-closed family of $\mathbb{F}$-representable matroids.
Then there are only finitely many $\mathbb{F}$-representable matroids $M$ such that $M \notin F$ and any minor of $M$ is in $F$.
In other words, for each finite field $\mathbb{F}$ and each minor-closed class of $\mathbb{F}$-representable matroids, there are only finitely many $\mathbb{F}$-representable excluded minors.
\end{thm}

This is an extension of WQO theorem from graphs to $\mathbb{F}$-representable matroids.
