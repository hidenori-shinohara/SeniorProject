\section{The introduction to Rota's Conjecture}
This chapter introduces Rota's Conjecture and an outline of how the proof goes.

\begin{conj}[Rota's Conjecture]
For each finite field $\mathbb{F}$, there are, up to isomorphism, only finitely many excluded minors for the class of $\mathbb{F}$-representable matroids.
\end{conj}(From \textit{Solving Rota's conjecture})

In other words, given a finite field $\mathbb{F}$, let $F$ be a family of all matroids that are representable over $\mathbb{F}$.
By the theorem from the previous chapter, we know that $F$ is minor-closed, i.e., any minor of any element in $F$ is in $F$.
Then, an excluded minor is a matroid $M \notin F$ such that any minor of $M$ is in $F$.
The conjecture states that for any finite field $\mathbb{F}$, there are only finitely many excluded minors.

First, we will introduce some results related to excluded minors.


\begin{thm}
The Fano matroid is an excluded minor for the class of $\mathbb{F}$-representable matroids if the characteristic of $\mathbb{F}$ is not 2.
\end{thm}
\begin{proof}
Let $M$ be the Fano matroid.
We know from the previous chapter that $M$ is not representable in a field if the characteristic of the field is not 2.
Therefore, it suffices to show that any minor of $M$ is representable in those fields.
Since any minor of a representable matroid is always representable, it suffices to show that $M / e$ and $M \backslash e$ are representable for any $e$ in those fields.
By the symmetry of the Fano matroid(this needs a bit stronger reasoning), it suffices to show that $M / 7$ and $M \backslash  7$ are both representable in those fields.
$M \backslash 7$ is isomorphic to the column matroid of the following matrix:
$\begin{pmatrix}
1 & 0 & 0 & 1 & 0 & 1 \\
0 & 1 & 0 & 1 & 1 & 0 \\
0 & 0 & 1 & 0 & 1 & -1 \\
\end{pmatrix}$, where -1 is the additive inverse of 1.
Similarly, $M / 7$ is isomorphic to the column matroid of the following matrix:
$\begin{pmatrix}
1 & 0 & 1 & 1 & 1 & 0 \\
0 & 1 & 1 & 1 & 0 & 1
\end{pmatrix}$.
\end{proof}

Geelen, Gerards, Whittle announced that they have solved Rota's Conjecture.
Here is a brief summary of their proof.

As matroid theory is, in a way, an abstraction of graph theory, they have used several theorems from graph theory.
Particularly, they extended Graph Minors Project to matroids.
In "Solving Rota's Conjecture", they mention that this extension was really important since "the Graph Minors Theory had exactly the kind of general purpose tools that we lacked".
Note that the minor of a graph can be obtained by deleting edges and vertices and by contracting edges.

\begin{thm}
Let $F$ be a minor-closed family of graphs, that is, $\forall G \in F$, any minor of $G$ is in $F$.
Then there are only finitely many graphs $H$ such that $H \notin F$ and any minor of $H$ is in $F$.
In other words, each minor-closed class of graphs has only finitely many excluded minors.
\end{thm}

This is called the graph minor theorem.
One special case of this theorem is a set of planar graphs.
Any minor of a planar graph is also planar.
Therefore, a set $S$ of planar graphs is minor-closed.
Kuratowski's theorem indeed states that there are only two excluded minors, $K_{3, 3}, K_5$.

Here is the extension of the graph minor theorem to matroids.

\begin{thm} [Matroid WQO Theorem]
Let a finite field $\mathbb{F}$ be given and $F$ be a minor-closed family of $\mathbb{F}$-representable matroids.
Then there are only finitely many $\mathbb{F}$-representable matroids $M$ such that $M \notin F$ and any minor of $M$ is in $F$.
In other words, for each finite field $\mathbb{F}$ and each minor-closed class of $\mathbb{F}$-representable matroids, there are only finitely many $\mathbb{F}$-representable excluded minors.
\end{thm}

More precisely, this is an extension from graphs to $\mathbb{F}$-representable matroids.
(So, basically is this a theorem in Linear Algebra because $\mathbb{F}$-representable matroids are basically matrices?)

Although the Matroid WQO Theorem and the graph minor theorem both discuss excluded minors, the Matroid WQO Theorem does not imply Rota's Conjecture.
Rota's Conjecture is about a family of representable matroids, so all the excluded minors are non-representable.
In other words, obviously, there are never any $\mathbb{F}$-representable excluded minors.

They mention that "the most significant ingredient is an analogue of the Graph Minors Structure Theorem".

(Put some description once I understand what Graph Minors Structure Theorem is)

Another important concept is connectivity.
\begin{defn}
A $k$-separation in a matroid $M = (E, \mathcal{I})$ is a partition $(X, Y)$ of $E$ such that $r_M(X) + r_M(Y) - r_M(E) < k$ and $\lvert X \rvert, \lvert Y \rvert \geq k$.
\end{defn}
\begin{defn}
A matroid is $k$-connected if it has no $l$-separation for any $l < k$.
\end{defn}

\begin{lem}
For each field $\mathbb{F}$, each excluded minor for the class of $\mathbb{F}$-representable matroids is 3-connected.
\end{lem}


There are a few basic results about this.

\begin{thm}
No matroid has 0-separation.
\end{thm}
\begin{proof}
Let $I$ be a maximal independent set.
Then $\lvert I \rvert = r_M(E)$.
Let $I_X = I \cap X, I_Y = I \cap Y$.
Then $\lvert I_X \rvert \leq r_M(X), \lvert I_Y \rvert \leq r_M(Y)$.
Therefore, $r_M(X) + r_M(Y) - r_M(E) \geq 0$
\end{proof}


\begin{thm}
$U_{2, 4}$ is indeed 3-connected.
\end{thm}
$U_{2, 4}$ is the only forbidden minor in $GF(2)$.
\begin{proof}
Let a partition of $E$, $(X, Y)$, be given.
Without loss of generality, $\lvert X \rvert \leq \lvert Y \rvert$.
Consider the case when $\lvert X \rvert = 1$.
Then we know that $r_M(X) = 1, r_M(Y) = 2$.
Consider the case when $\lvert X \rvert = 2$.
Then we know that $r_M(X) = 2, r_M(Y) = 2$.
Since $r_M(X) + r_M(Y) - r_M(E) \geq 1$ for any $(X, Y)$, a 1-partition does not exist.
Since $r_M(X) + r_M(Y) - r_M(E) < 2$ only if $r_M(X) = 1$, a 2-partition does not exist either.
Therefore, $U_{2, 4}$ is indeed 3-connected.
\end{proof}

A weak version of this lemma can be proved as following:

\begin{lem}
Let a finite field $\mathbb{F}$ and a matroid $M$ be given.
If $M$ has a 1-separation $(X, Y)$, then $M$ is not an excluded minor.
In other words, either it is representable or some of its minor is unrepresentable.
\end{lem}

\begin{proof}
$r_M(X) + r_M(Y) \geq r_M(E)$ since a maximal independent set must be split into $X, Y$.
Therefore, the existence of 1-separation implies that $r_M(X) + r_M(Y) = r_M(E)$.
Without loss of generality, $\lvert X \rvert \leq \lvert Y \rvert$.
Suppose $\lvert X \rvert \geq 2$.
I think I can prove this, but I haven't thought this through...
\end{proof}

