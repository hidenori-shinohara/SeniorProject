\section{More discussion on matroid representability}
This chapter will introduce a new concept, a matroid minor, which is crucial when discussing the matroid representability.
Rota's conjecture will be also introduced at the end of the chapter.
(I am thinking of adding a sketch of the proof by the end of the semester if I can understand the outline)

In order to introduce a matroid minor, we first need to introduce two operations on matroids. Deletion and contraction.

\begin{defn}
Let $M = (E, \mathcal{I})$, $X \subseteq E$ be given.
$M \backslash X$ denotes a deletion of $X$ in $M$ and is defined to be $(E - X, \{ I \in \mathcal{I} \mid X \cap I = \emptyset \})$
\end{defn}

Note that $M \backslash e$ for some element $e \in E$ is equivalent to $M \backslash \{ e \}$.

\begin{thm}
Let $M = (E, \mathcal{I})$, $X \subseteq E$ be given.
$M \backslash X$ is indeed a matroid
\end{thm}

\begin{proof}
Let $M' = (E', \mathcal{I}') = M \backslash X$.
Since $\emptyset \in \mathcal{I}$ and $X \cap \emptyset = \emptyset$, $\emptyset \in \mathcal{I}$.
Let $I \in \mathcal{I}', J \subseteq I$. 
Since $I \in \mathcal{I}$, $J \in \mathcal{I}$.
Since $J \subseteq I$ and $I \cap X = \emptyset$, $J \cap X = \emptyset$.
Therefore, $J \in \mathcal{I}'$.
Let $A, B \in \mathcal{I}'$ such that $\lvert A \rvert < \lvert B \rvert$.
Since $A, B \in \mathcal{I}'$, we know that $A, B \in \mathcal{I}$.
Therefore, we can find $x \in B - A$ such that $(A \cup \{ x \}) \in \mathcal{I}$.
Since $(A \cup \{ x \}) \subseteq (A \cup B)$ and $X \cap A = X \cap B = \emptyset$, $X \cap (A \cup \{ x \}) = \emptyset$.
Therefore, $A \cup \{ x \} \in \mathcal{I}'$.
Thus, we have found such $x \in B - A$ that $A \cup \{ x \} \in \mathcal{I}'$.
Since $M' = M \backslash X$ follows the three properties, it is indeed a matroid. 
\end{proof}

\begin{defn}
Let $M = (E, \mathcal{I})$, $e \in E$ be given.
$M / e$ denotes contraction of $M$ by $e$ and 
$M / e = \begin{cases}
      M \backslash e, \text{if $e$ is a loop},\\
      (E - \{ e \}, \{ I \in \mathcal{I} \mid e \notin I, (I \cup \{ e \}) \in \mathcal{I}\}), \text{ otherwise}.
         \end{cases}$
\end{defn}


\begin{thm}
Contraction by an element indeed generates a matroid.
\end{thm}

\begin{proof}
If $e$ is a loop, $M / e$ is obviously a matroid since we know that deletion always generates a matroid.
Suppose otherwise.
Let $\mathcal{I}'$ denote the independent sets of $M / e$.
First, $\emptyset \in \mathcal{I}, e \notin \emptyset$. Since $e$ is not a loop, $(\emptyset \cup \{ e \}) \in \mathcal{I}$.
Therefore, $\emptyset \in \mathcal{I}'$.
Let $I \in \mathcal{I}', J \subseteq I$.
Since $I \in \mathcal{I}$, $J \in \mathcal{I}$.
Since $e \notin I$, $e \notin J$.
Since $(I \cup \{ e \}) \in \mathcal{I}$ and $J \subseteq I$, $(J \cup \{ e \}) \in \mathcal{I}$.
Therefore, $J \in \mathcal{I}'$.
Let $A, B \in \mathcal{I}'$ such that $\lvert A \rvert < \lvert B \rvert$.
Let $A' = A \cup \{ e \}, B' = B \cup \{ e \}$.
Since $A, B \in \mathcal{I}'$, $A', B' \in \mathcal{I}$.
Since $e \notin A, e \notin B$, $\lvert A' \rvert < \lvert B' \rvert$.
Let $x \in B' - A'$ such that $A' \cup \{ x \} \in \mathcal{I}$.
Since $B' - A' = B - A$, $x \in B - A$.
For such $x$, we just showed that $A \cup \{ e \} \cup \{ x \} \in \mathcal{I}$.
Also, $x \neq e$ since $e \in A'$.
Therefore, $A \cup \{ x \} \in \mathcal{I}'$. 
Hence, we have found $x \in B - A$ such that $A \cup \{ x \} \in \mathcal{I}'$.
Since this follows three properties given in the definition, this is indeed a matroid.
Therefore, contraction by an element indeed generates a matroid.
\end{proof}


Here are a few simple yet useful results about deletion.

\begin{thm}
Let a matroid $M = (E, \mathcal{I}), X \subseteq E$ be given.
Let $A$ be an independent set in $M \backslash X$.
Then $A$ is independent in $M$.
\end{thm}

\begin{proof}
The independent sets of $M \backslash X$ is $\{ I \in \mathcal{I} \mid (I \cap X) = \emptyset \}$.
It is easy to see that it is a subset of $\mathcal{I}$.
Since $A$ is in the subset of $\mathcal{I}$, $A$ must be in $\mathcal{I}$.
\end{proof}

In other words, this means that deletion never "adds" a new element to the independent sets.


\begin{thm}
The deletion of a loop does not change the independent sets.
\end{thm}

\begin{proof}
Let $M = (E, \mathcal{I}), e \in E, \{ e \} \notin \mathcal{I}$.
The independent sets of $M \backslash e$ is $\{ I \in \mathcal{I} \mid (I \cap \{ e \}) = \emptyset \}$.
Since $\{ e \}$ is a loop, no independent set can contain $e$.
Therefore, the independent sets of $M \backslash e$ is identical to $\mathcal{I}$.
\end{proof}

Now that we have defined contraction of a matroid by an element, we can define contraction by a subset of a ground set.
\begin{defn}
Let $M = (E, \mathcal{I}), X = \{ x_1, \cdots, x_k \} \subseteq E$.
$M / X$ is defined to be $(((M/x_1)/x_2) \cdots)/x_k)$.
\end{defn}

It is not obvious that this is well-defined. 
In other words, it is not obvious that the order of contraction does not matter.
The following theorem shows that the order does not matter.

\begin{thm}
For any given matroid $M = (E, \mathcal{I})$,
$(M / e) / f = (M / f) / e$ for any $e \neq f \in E$.
\end{thm}

\begin{proof}
There are a few cases.
\begin{enumerate}

\item $e, f$ are both loops. \\
$(M / e) / f = (M \backslash e) / f$
Since deletion of a loop does not change the independent sets, $f$ is a loop in $(M \backslash e)$.
Therefore, $(M / e) / f = (M \backslash e) \backslash f$.
Again, deletion of $f$ does not change the independent sets since $f$ is a loop in $(M / e)$.
Therefore, we have $(M / e) / f = (E - \{ e, f \}, \mathcal{I})$.
By symmetry, $(M / e) / f = (M / f) / e$.

\item 
One of $e, f$ is a loop, and the other one is not. \\
Without loss of generality, assume $e$ is a loop.
$(M / e) / f = (M \backslash e) / f$.
Since deletion of a loop does not change the independent set, the independent set of $(M / e)$ is $\mathcal{I}$.
Therefore, the independent sets of $(M / e) / f$ is $\mathcal{I}' = \{ I \in \mathcal{I} \mid f \notin I, (I \cup \{ f \}) \in \mathcal{I} \}$.
On the other hand, it is easy to see that $\mathcal{I}'$ is identical to the independent sets of $(M / f)$.
Since contraction by an element does not add new elements to the independent sets, $e$ is a loop in $(M / f)$.
Since deletion by a loop does not change the independent sets, the independent sets of $(M / f) / e$ is $\mathcal{I}'$.
Now we confirmed that $(M / e) / f$ and $(M / f) / e$ have the same independent sets.
Therefore, $(M / e) / f = (M / f) / e$.

\item Neither of them is a loop, and $\{ e, f \} \in \mathcal{I}$. \\
The independent set of $M / e$ is $\mathcal{I}' = \{ I \in \mathcal{I} \mid e \notin I, (I \cup \{ e \}) \in \mathcal{I} \}$.
Since $\{ e, f \} \in \mathcal{I}$, $\{ f \} \in \mathcal{I}'$. 
Therefore, $f$ is not a loop in $M / e$.
Hence, the independent sets of $(M / e) / f$ is $\mathcal{I}'' = \{ I \in \mathcal{I}' \mid f \notin I, (I \cup \{ f \}) \in \mathcal{I}' \}$.
$\mathcal{I}''$ is actually equivalent to $S = \{ I \in \mathcal{I} \mid e \notin I, f \notin I, (I \cup \{ e, f \}) \in \mathcal{I} \}$.
We can prove $\mathcal{I}'' = S$ by starting to show that $\mathcal{I}'' \subseteq S$.
Let $I \in \mathcal{I}''$.
Since $I$ is an independent set of $(M / e) / f$, we know that $e, f \notin I$.
Since $(I \cup \{ f \}) \in \mathcal{I}'$, we also know that $((I \cup \{ f \}) \cup \{ e \}) \in \mathcal{I}$.
Therefore, $(I \cup \{ e, f \}) \in \mathcal{I}$.
Thus $I \in S$, and $\mathcal{I}'' \subseteq S$.
Now, we want to show that $S \subseteq \mathcal{I}''$. 
Let $I \in S$. By the definition of $S$, we know that $e, f \notin I$.
Since $(I \cup \{ e, f \}) \in \mathcal{I}$, we know that $(I \cup \{ e \}) \in \mathcal{I}$.
Since $((I \cup \{ f \}) \cup \{ e \}) \in \mathcal{I}$ and $e \notin (I \cup \{ f \})$, we know that $(I \cup \{ f \}) \in \mathcal{I}'$.
Since $f \notin I$ and $(I \cup \{ f \}) \in \mathcal{I}'$, $I \in \mathcal{I}''$.
Hence, $S \subseteq \mathcal{I}''$.

Combining these two results, we know that $S = \mathcal{I}''$.
By the symmetry, $(M / e) / f$ and $(M / f) / e$ have the same independent sets.
Therefore $(M / e) / f = (M / f) / e$.

\item Neither of $e, f$ is a loop, but $\{ e, f \} \notin \mathcal{I}$.\\
The independent set of $M / e$ is $\mathcal{I}' = \{ I \in \mathcal{I} \mid e \notin I, (I \cup \{ e \}) \in \mathcal{I} \}$.
Since $\{ e, f \} \notin \mathcal{I}$, $f$ is a loop in $M / e$.
Therefore, $(M / e) / f = (M / e) \backslash f$.
Since the deletion of a loop does not change the independent sets, the independent sets of $(M / e) / f$ is $\mathcal{I}'$.
By applying the same argument, the independent set of $(M / f) / e$ is $\mathcal{I}'' = \{ I \in \mathcal{I} \mid f \notin I, (I \cup \{ f \} ) \in \mathcal{I} \}$.
We want to show that $\mathcal{I}'  = \mathcal{I}''$.
By the symmetry, it suffices to show that $\mathcal{I}' \subseteq \mathcal{I}''$.
Let $I \in \mathcal{I}'$.
Since $\{ e, f \}$ is dependent, $f \notin I$. (Otherwise, $I \cup \{ e \}$ would be dependent.)
Since both $(I \cup \{ e \})$ and $\{ f \}$ are independent, we can grow $\{ f \}$ by adding elements from $(I \cup \{ e \})$ until they have the same size.
Since $\{ e, f\}$ is dependent, we never add $e$.
In other words, we add every element from $\mathcal{I}$.
It means that $I \cup \{ f \}$ is independent.
Therefore, $I \in \mathcal{I}''$, and thus $\mathcal{I}' \subseteq \mathcal{I}''$.
By symmetry, $\mathcal{I}'' \subseteq \mathcal{I}'$.
Therefore, $\mathcal{I}' = \mathcal{I}''$.
\end{enumerate}

Therefore, in any case, $(M / e) / f = (M / f) / e$.
\end{proof}


%\item Maybe talk about dual matroid
%  \begin{itemize}
%    \item redefine contraction/deletion in terms of dual
%  \end{itemize}

Now that we have defined contraction and deletion, we can define a \textit{minor} of a matroid.

\begin{defn}
A minor of a matroid is a matroid that can be obtained by some (possibly zero) a number of contraction and deletion.
\end{defn}

Therefore, most matroids have more than one minors.
To define the matroid minor more concisely, we will prove the following theorem.


\begin{thm}
Let a matroid $M = (E, \mathcal{I})$ and $e \neq f \in E$ be given.
Then $(M / e) \backslash f = (M \backslash f) / e$.
\end{thm}

\begin{proof}
Prove it here.
\end{proof}

By this theorem, we know that any series of operations can be expressed as $(M / A) \backslash B$ where $A, B$ are disjoint subsets of $E$.
Therefore, the following definition is equivalent to the previous definition.

\begin{defn}
Let $M = (E, \mathcal{I})$ be given.
Let $A, B$ be disjoint subsets of $E$.
Then a matroid $(M / A) \backslash B$ is called a minor of $M$.
\end{defn}

Moreover, if $A \cup B \neq \emptyset$, we call $(M / A) \backslash B$ a \textit{proper minor}.

Of course, we could have defined a minor as $(M \backslash A) / B$ instead of $(M \ A) \backslash B$.


\subsection{What do contraction and deletion mean in graphs and vector spaces?}
In vector spaces, contraction can be considered as projection.
Maybe insert a figure.

\subsection{Why do these matter?}
The discussion of contraction and deletion is very important when discussing the representability of matroids since if a matroid is representable over some field $\mathbb{F}$, its minor is always representable over $\mathbb{F}$.

\begin{thm}
Let a matroid $M = (E, \mathcal{I})$ such that it is representable over $\mathbb{F}$.
Any minor of $M$ is representable over $\mathbb{F}$.
\end{thm}

\begin{proof}
It suffices to show that $M / e$ and $M \backslash e$ are both representable over $\mathbb{F}$ for any $e \in E$.
Let $r$ be a rank of $M$, $n = \lvert E \rvert$.
Let $e \in E$ be given.
Let $A = \begin{pmatrix}u_1 u_2 \cdots u_n\end{pmatrix} \in \mathbb{F}^{r \times n}$ be a matrix such that the column matroid of $M$ is isomorphic to $A$.
Without loss of generality, we can assume $E = \{ 1, 2, \cdots, n \}$ and $e = n$.
\begin{enumerate}
\item 
  First, we prove the case of deletion.
  We claim that $M \backslash e$ is isomorphic to the column matroid of $\begin{pmatrix} u_1 u_2 \cdots u_{n-1} \end{pmatrix}$.
  We prove so by comparing the independent sets of each matroid.
  By definition, $M \backslash e = (E - \{ e \}, \{ I \in \mathcal{I} \mid e \notin I \})$.
  Let $U = \{ u_{i_1}, u_{i_2}, \cdots, u_{i_k} \}$ be a subset of $\{ u_1, u_2, \cdots, u_{n-1} \}$.
  We want to show that $U$ is linearly independent if and only if $\{ i_1, i_2, \cdots i_k \}$ is linearly independent in $M \backslash e$.
  Suppose $U$ is linearly independent. 
  Then $\{ i_1, i_2, \cdots i_k \}$ is independent in the column matroid of $A$.
  Since $\{ i_1, i_2, \cdots i_k \}$ is in $\mathcal{I}$ and does not contain $e = n$, it is independent in $M \backslash e$ as well.
  Suppose $U$ is linearly dependent. 
  Then $\{ i_1, i_2, \cdots i_k \}$ is dependent in the column matroid of $A$.
  Since $\{ i_1, i_2, \cdots i_k \}$ is not in $\mathcal{I}$ it is dependent in $M \backslash e$ as well.
  Therefore, $M \backslash e$ is representable over $\mathbb{F}$.
\item
  Next, we prove the case of contraction.
  We prove so by comparing the independent sets of each matroid.
  By definition, $M / e = (E - \{ e \}, \{ I \in \mathcal{I} \mid e \notin I, (e \cup I) \in \mathcal{I}\})$.
  We claim that $M / e$ is isomorphic to the column matroid of $B \in \mathbb{F}^{r \times n-1}$, 
  where $i$th column of $B$, $b_i$, is $\displaystyle u_i - \frac{u_i \cdot u_n}{\lvert u_n \rvert ^ 2} u_n$.
\end{enumerate}
\end{proof}

However, neither the converse nor the inverse of this theorem is always true.
Any matroid has a representable minor since $U_{0, k}$ is a minor of any matroid.
Also, $U_{2, 4}$ is not a binary matroid, but any minor of it only contains at most 3 elements, so we know that any minor of $U_{2,4}$ is regular by the theorem.
Rota's conjecture is about unrepresentable matroids any of whose minor is representable.

\subsection{The introduction of Rota's conjecture}
\begin{conj}
For any finite field $\mathbb{F}$, there are only finitely many unrepresentable matroids all of whose minors are representable. 
In other words, there are only finitely many \textit{excluded minors}.
\end{conj}

(Add a simple sketch of proof if I understand it)
