\section{Introduction to Matroid}

A matroid is a mathematical structure that abstracts the concept of linear independence.
A linearly independent set of vectors has a lot of good properties.
Interestingly, a lot of sets of other mathematical objects often share those properties.
For example, any subset of a linearly independent set of vectors is always linearly independent.
Consider a subset of edges of a graph that does not contain any cycle.
Then any subset of it does not contain any cycle.
The matroid theory is an attempt to mathematically formalize those properties and investigate in them.

\begin{defn}
A matroid $M = (E, \mathcal{I})$ is a pair such that $E$ is a finite set of elements, and $\mathcal{I}$ is a family of subsets of $E$ with the following properties:
\begin{itemize}
\item $\emptyset \in \mathcal{I}$
\item For any $A\in \mathcal{I}$, any subset of $A$ is in $\mathcal{I}$.
\item For any $A, B \in \mathcal{I}$ such that $\lvert A \rvert < \lvert B \rvert$, there always exists $x \in B - A$ such that $A \cup \{ x \} \in \mathcal{I}$.
\end{itemize}
\end{defn}
$E$ is called the ground set, $\mathcal{I}$ is called the independent sets. A subset of $E$ is called independent if and only if it is in $\mathcal{I}$.
In this paper, we will put our focus on a matroid with a finite ground set. 


There are some basic matroids that are important in the following discussions.
We will start by introducing a column matroid. 
A column matroid is constructed from a matrix over a field $\mathbb{F}$.

\begin{defn}
Let a matrix $A$ with $m$ rows over $\mathbb{F}$ be given.
A column matroid $M$ of $A$ is a matroid with a ground set $\{ 1, 2, \cdots m \}$.
A subset of $E$ is independent in $M$ if and only if the set of the corresponding column vectors is linearly independent.
\end{defn}

\begin{thm}
A column matroid is indeed a matroid.
\end{thm}

\begin{proof}
First, an empty set of vectors is linearly independent by definition.
Suppose there exists a linearly independent set of vectors that has a linearly dependent subset.
Let $A = \{a_1, \cdots, a_k\} \subseteq B = \{b_1, \cdots, b_n\}$ be such sets.
Since $A$ is linearly dependent, there exist constants $c_1, \cdots, c_k$ such that $c_1 a_1 + \cdots + c_k a_k = 0$ and not all $c_i$'s are $0$.
However, that implies that we can find constants $d_1, \cdots, d_n$ such that $d_1 b_1 + \cdots + d_k b_k = 0$ and not all $d_i$'s are $0$.
This is a contradiction since $B$ is supposed to be independent.
Therefore, such subset must not exist, and thus all subsets of linearly independent sets are linearly independent.
Now we want to prove the third property.
Let $A = \{a_1, \cdots, a_k\}, B = \{b_1, \cdots, b_n\}$ be linearly independent sets, and assume that $k < n$.
Suppose for each $i = 1, \cdots, n$, $b_i \in span\{ a_1, \cdots, a_k \}$.
It means that the span of $B$ is a subspace of the span of $A$, which has a smaller dimension than $n$.
That is a contradiction.
Therefore, there exists $i$ such that $b_i \notin span \{ a_1, \cdots, a_k \}$.
For such $i$, $\{ a_1, \cdots, a_k, b_i \}$ should be linearly independent.
Since a column matroid satisfies the three properties, it is indeed a matroid.
\end{proof}


\begin{defn}
A uniform matroid $U_{r, k}$ is a matroid such that $E = \{ 1, \cdots, k \}$ and $\mathcal{I} = \{ X \mid  X \subseteq E, \lvert X \rvert \leq r \}$.
\end{defn}

It is easy to see that a uniform matroid is indeed a matroid.

% If I need more pages, just introduce a graph matroid

Here are some important results that will show up later in this paper.

\begin{thm}
All maximal independent sets have the same size.
\end{thm}

\begin{proof}
Let $X, Y$ be maximal independent sets of some matroid.
Suppose $\lvert X \rvert \neq \lvert Y \rvert$.
Without loss of generality, $\lvert X \rvert < \lvert Y \rvert$.
By the third property of a matroid, there exists $e \in Y - X$ such that $X \cup \{ e \}$ is independent.
It is a contradiction since $X$ is a maximal independent set.
Therefore $\lvert X \rvert = \lvert Y \rvert$.
\end{proof}

This is indeed true in linear algebra.
Given a matrix, any maximal independent subset of column vectors always has the same size. In linear algebra, this number is often referred to as the \textit{dimension} of a vector space or the \textit{rank} of a matrix.
In the matroid theory, we use the term \textit{rank} as well.


\begin{defn}
The rank of a matroid is the size of a maximal independent set.
\end{defn}

The rank of a column matroid is equal to the rank of the matrix since the subset of a ground set is independent if and only if the subset of column vectors is linearly independent.

\begin{defn}
Let $M = (E, \mathcal{I})$ be given.
$e \in E$ is called a loop if $\{ e \} \notin \mathcal{I}$.
\end{defn}

We will conclude this chapter by introducing the notion of isomorphic matroids.
\begin{defn}
Let $M_1 = (E_1, \mathcal{I}_1), M_2 = (E_2, \mathcal{I}_2)$ be given.
$M_1, M_2$ are isomorphic to each other if there exists a bijective mapping $\phi: E_1 \rightarrow E_2$ such that
$\forall X \subseteq E_1, X \in \mathcal{I}_1 \iff \{ \phi(e) : e \in X \} \in \mathcal{I}_2$.
\end{defn}

Two isomorphic matroids have the same structure.
