\section{Introduction to Matroid}

A matroid is a mathematical structure that abstracts the concept of linearly independence.
Linearly independent sets of vectors have a lot of good properties.
Interestingly, a lot of sets of other mathematical objects often share those properties.
For example, any subset of a linearly independent set of vectors is always linearly independent.
Consider a subset of edges of a graph that does not contain any cycle.
Then any subset of it does not contain any cycle.
The matroid theory is an attempt to mathematically formalize those properties and investigate in them.

\begin{defn}
A matroid $M = (E, \mathcal{I})$ is a pair such that $E$ is a finite set of elements, and $\mathcal{I}$ is a family of subsets of $E$ with the following properties:
\begin{itemize}
\item $\emptyset \in \mathcal{I}$
\item For any $A\in \mathcal{I}$, any subset of $A$ is in $\mathcal{I}$.
\item For any $A, B \in \mathcal{I}$ such that $\lvert A \rvert < \lvert B \rvert$, there always exists $x \in B - A$ such that $A \cup \{ x \} \in \mathcal{I}$.
\end{itemize}
$E$ is called the ground set, $\mathcal{I}$ is called the independent sets. A subset of $E$ is called independent if and only if it is in $\mathcal{I}$.
\end{defn}

In this paper, we will put our focus on a matroid with a finite ground set.

There are some basic matroids that are important in the following discussions.

We will start by introducing a column matroid. 
A column matroid is constructed from a matrix over a field $\mathbb{F}$.

\begin{defn}
Let a matrix $A$ with $m$ rows over $\mathbb{F}$ be given.
A column matroid $M$ of $A$ is a matroid with a ground set $\{ 1, 2, \cdots m \}$.
A subset of $E$ is independent in $M$ if and only if the set of column vectors corresponding to it is linearly independent.
\end{defn}

\begin{proof}
Prove it!
\end{proof}


\begin{defn}
A uniform matroid $U_{r, k}$ is a matroid such that $E = \{ 1, \cdots, k \}$ and $\mathcal{I} = \{ X \mid  X \subseteq E, \lvert X \rvert \leq r \}$.
\end{defn}

It is easy to see that a uniform matroid is indeed a matroid.

